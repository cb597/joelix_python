\begin{aufg}
  In dieser Aufgabe wollen wir Klassen als Enten interpretieren, wenn sie sich wie Enten verhalten.
  Dazu verwenden wir die Klasse \lpy{Ente}:
  \begin{lstlisting}
class Ente:
  """Eine einfache Entenklasse"""
  def __init__(self, quaks=1):
    """Erstellt eine Ente.
       Die Anzahl der quaks kann uebergeben werden."""
    self.quaks = quaks
  
  def __str__(self):
    return self.quaks * 'Quack!'
  \end{lstlisting}
  
  Erweiter die Klasse \lpy{Polynom} um die spezielle Memberfunktion \lpy{__ente__},
  die eine Ente zurückgibt, wobei die Anzahl der Quaks dem Grad des Polynoms entsprechen soll.
  
  Schreibe eine Funktion \lpy{ente(arg)}, die sich genau wie die bekannten Funktionen \lpy{str(arg)} oder \lpy{int(arg)} verhält.
  Das heißt, folgender Code soll interpretierbar sein:
  \begin{lstlisting}
p = Polynom(...) # Erstellt Polynom p vom Grad n
e = ente(p)      # Interpretiert p als Ente
print(e)         # Druckt: Quack!Quack!...Quack! (n mal)
  \end{lstlisting}

\end{aufg}
