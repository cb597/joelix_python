\begin{aufg}
  Implementiere eine Klasse \lpy{Polynom}, welche ein Polynom mit reellen 
  Koeffizienten modelliert.
  Die Koeffizienten des Polynoms sollen dabei dem Konstruktor übergeben werden
  und innerhalb der Klasse als Tupel gespeichert werden.
  Implementiere die grundlegenden Rechenfunktionen auf Polynomen, wie 
  Addition, Substraktion und Multiplikation, sowie die Multiplikation eines 
  Polynoms mit einem \lpy{float}.
  Außerdem sollen Polynome eine Gradfunktion besitzen und als String interpretiert werden können.

  Mache dabei intensiv Gebrauch von den speziellen Memberfunktion, die du
  in der Vorlesung kennengelernt hast und die im Skript genauer besprochen werden.
\end{aufg}
