\section{\texorpdfstring{Python + C = $\heartsuit$}{Python plus C gleich Liebe}}
\label{section:python_mit_c}
Wir nutzen \CC für Programme, die (so gut wie) ausschließlich zeitkritische Probleme lösen sollen (z.B.\ einen Poissonlöser \cite{joelixC}) oder die direkten Hardwarezugriff brauchen (z.B.\ Treiber).
Für alles andere nutzen wir \Python.
Es gibt mehrere Möglichkeiten die Vorteile beider Sprachen zu kombinieren.
Hier stellen wir zwei Möglichkeiten vor.
Bei beiden bindet man in \C geschriebenen Bibliotheken in ein \Python Projekt ein.
Die sogenannten ``Extension Modules'' sind nicht so schnell zu coden wie die sogenannten ``CTypes'', haben aber eine höhere Performance als letztere.
Die ``CTypes'' sind nicht so performant wie ``Extension Modules'', sind dafür aber schneller zu coden.

\subsection{Extension Modules}
\label{section:python_mit_c:extension_modules}
Bevor wir erklären, was ``Extension Modules'' sind und wie sie in \Python eingebunden werden können, machen wir darauf aufmerksam,
dass Extension Modules nur mit dem \Python-Interpreter CPython funkionieren:
``The C extension interface is specific to CPython, and extension modules do not work on other Python implementations.'',
see \cite[Extending Python with C or C++; \S 1]{Python3, Python2}.
Das liegt daran, wie Extension Modules technisch umgesetzt sind.
Der \Python-Interpreter CPython ist ein in \C geschriebenes Programm.
So wie alle in \C geschriebenen Programme, kann auch CPython in \C geschriebene Bibliotheken zur Laufzeit nachladen und benutzen.
Ein Extension Module ist (rein technisch gesehen) eine \C Bibliothek, die CPython zur Laufzeit nachladen und benutzen kann.

Nun besprechen wir, wann man Extension Modules verwenden sollte und wie man sie verwendet.

\subsubsection{Die Ausgangssituation}
Gehen wir davon aus, dass wir in einem \Python Projekt eine Funktion \lpy{addition_py} haben, die besonders schnell ausgeführt werden muss oder Hardwarezugriff braucht.
Der Einfachheit halber wollen wir davon ausgehen, dass \lpy{addition_py} eine feste Anzahl von Parametern eines festen Typs benötigt und einen definierten Ausgabetyp hat.

\subsubsection{Schnelle C-Funtkion implementieren}
Als erstes implementieren wir eine in \C geschriebenene Funktion \lcpp{addition}, die sich wie die Funktion \lpy{addition_py} verhält.
Außerdem binden wir die \lcpp{<Python.h>} ein, die vom \Python-Standard bereit gestellt wird%
\footnote{Es kann sein, dass man die Entwicklerbibliotheken nachträglich installieren muss.
Unter Debianderivaten geschieht dass mit \lstinline[language=bash]|sudo apt-get install python3-dev|.}.
Es ist wichtig, dass \lcpp{<Python.h>} vor allen anderen Bibliotheken einbegunden wird,
denn \lcpp{<Python.h>} stellt gewissen Makros bereit, welche die anderen eingebundenen Bibliotheken unter Umständen modifiziert.
Unser Beispiel sieht also momentan so aus:
\begin{lstlisting}[language=C++, style=CPP]
// Datei: addition.c
// Die Python.h muss vor allen anderen Bibliotheken eingebunden werden.
#include <Python.h>

int addition( int op1, int op2 )
{
  return op1 + op2;
}
\end{lstlisting}

\subsubsection{Jede schnelle C-Funktion braucht eine Hilfsfunktion}
Nun erstellen wir (immer noch in \C) für jede schnelle \C-Funktion genau eine Hilfsfunktionen.
Mithilfe der Hilfsfunktionen kann der (ebenfalls in \C geschriebenene) \Python-Interpreter CPython unsere Funktion \lcpp{addition} innerhalb unseres \Python Projekts aufrufen.
Wir erklären nun, wie die Hilfsfunktion aufgebaut ist.

(1) Die Signatur einer jeden Hilfsfunktionen ist:
\begin{lstlisting}[language=C++, style=CPP]
static PyObject* hilfsfunktionsname( PyObject* self, PyObject* args )
\end{lstlisting}

(2) Nun definiert man alle \C-Variablen, die wir für den Aufruf unserer \C-Funktion brauchen.
Das beinhaltet eine Variable für den Rückgabewert unserer \C-Funktion.
\begin{lstlisting}[language=C++, style=CPP]
static PyObject* add_hilfsfkt( PyObject* self, PyObject* args )
{
  // Erstelle fuer Rueckgabe und jeden Parameter eine Variable.
  int op1, op2, reuckgabe;
  // Hier gehts dann gleich weiter
  // ...
}
\end{lstlisting}

(3) Dann interpretieren wir die in \Python übergebenen Parameter als \C-Variablen.
Dazu benutzen wir die \C-Funktion
\begin{lstlisting}[language=C++, style=CPP]
int PyArg_ParseTuple( PyObject* args, const char* format, ... )
\end{lstlisting}
Was das erste Argument ist, müssen wir hier nicht nicht verstehen.
Das zweite Argument ist ein (konstanter) String, der erklärt, wie die übergebenen \Python-Parameter in \C-Variablen konvertiert werden sollen.
Dieses Konzept sollte von der \C-Funktion \lcpp{printf} bekannt sein.
Wir behandeln hier nur den einfachsten Fall, i.e.\ der \Python-Funktion werden nur die folgenden Typen übergeben: \lpy{int}, \lpy{float} oder \lpy{str}.
Wie man andere \Python-Objekte zu \C-Variablen konvertiert und wie man mit variablen Parametern umgeht behandeln wir hier nicht und verweisen auf \cite[Extending Python with C or C++; \S 1.7]{Python3}.
Hat die \Python-Funktion genau $k$ Parameter \lpy{p_1,...,p_k} und soll der $i$-te Parameter als \lpy{int}, \lpy{float} bzw.\ \lpy{str} interpretiert werden,
so erstellen wir einen String der Länge $k$, dessen $i$-tes Zeichen \lpy{'i'}, \lpy{'f'} bzw.\ \lpy{'z'} ist.
Im Anschluss übergeben wir der \C-Funktion \lcpp{PyArg_ParseTuple} genau $k$ Referenzen auf die Objekte, die wir im vorherigen Schritt erstellt haben.
In unserem Beispiel sieht die Hilfsfunktion bis jetzt also so aus:
\begin{lstlisting}[language=C++, style=CPP]
static PyObject* add_hilfsfkt( PyObject* self, PyObject* args )
{
  // Erstelle fuer Rueckgabe und jeden Parameter eine Variable.
  int op1, op2, reuckgabe;
  // Wandel Python-Variablen in C-Variablen um.
  PyArg_ParseTuple( args, "ii", &op1, &op2 )
  // Hier gehts dann gleich weiter
  // ...
}
\end{lstlisting}
Die Funktion \lcpp{PyArg_ParseTuple} gibt genau dann \lcpp{0} zurück, wenn die Konvertierung fehlgeschlagen hat.
In diesem Fall wollen wir die Hilfsfunktion ebenfalls abbrechen und geben \lcpp{NULL} zurück.

(4) Hat die Konvertietung der \Python-Parameter in \C-Variablen funktioniert, können wir unsere schnelle \C-Funktion mit den gegebenen Parametern aufrufen und den Rück\-ga\-be\-wert speichern.

(5) Nun konvertieren wir den Rückgabewert unserer schnellen \C-Funktion in ein \Python-Objekt und geben dieses als Rückgabewert der Hilfsfunktion zurück.
Die Konvertierung geschieht mit der \C-Funtkion \lcpp{Py_BuildValue( char* , ... )}.
Ähnlich wie oben gehen wir hier davon aus, dass der Rückgabewert unserer schnellen \C-Funktion der Typ \lcpp{int}, \lcpp{float} oder \lcpp{char*} ist.
Für andere Rückgabetypen verweisen wir auf \cite[Extending Python with C or C++; \S 1.9]{Python3}.
Der zu übergebe String ist \lstinline[language=C++,style=CPPinline]|"i"|, \lstinline[language=C++,style=CPPinline]|"f"| bzw.\ \lstinline[language=C++,style=CPPinline]|"s"|
je nachdem ob unser Rückgabetyp unserer schnellen \C-Funktion \lcpp{int}, \lcpp{float} oder \lcpp{char*} ist.

Damit sieht die fertige Hilfsfunktion für unser Beispiel so aus:
\begin{lstlisting}[language=C++, style=CPP]
static PyObject* add_hilfsfkt( PyObject* self, PyObject* args )
{
  // Erstelle fuer Rueckgabe und jeden Parameter eine Variable.
  int op1, op2, reuckgabe;
  // Wandel Python-Variablen in C-Variablen um.
  if( !PyArg_ParseTuple( args, "ii", &op1, &op2 ) ) return NULL;
  // Rufe die C-Funktion auf.
  reuckgabe = addition( op1, op2 );
  // Wandle C-Variable in Python-Variable um.
  return Py_BuildValue( "i", reuckgabe );
}
\end{lstlisting}

\subsubsection{Die Hilfsfunktion werden in einem Modul zusammengefasst}
Jetzt beschreiben wir ein Modul \lpy{hilfsmodul} welches am Ende in \Python eingebunden werden kann und unsere Funktionen bereit stellt.
Ein Modul bestehend aus einer sogenannten ``Method Table'' und einer ``Modulinitialisierung''.
Die Method Table ist ein Array von \lcpp{struct}s bestehend aus den vier Feldern:
Der Python-Name der Funktion;
der Funktionspointer auf die in \C implementierte Funktion;
der \lcpp{enum} der die Parameterübergabe beschreibt (siehe unten);
die Python-Beschreibung der Funktion.
Der Array wird mit \lstinline[language=C++,style=CPPinline]|{NULL, NULL, 0, NULL}| beendet.

Der besagte \lcpp{enum} beschreibt die Möglichkeiten wie die Parameter übergeben werden.
Wir konzentrieren uns hier einzig und allein auf den Wert \lcpp{METH_VARARGS} welcher der einfachen Übergabe von Variablen entspricht.
Die anderen Möglichkeiten Variablen zu übergeben sind in \cite[Extending Python with C or C++; \S 1.4]{Python3} beschrieben.

Die Method Table für unser Beispiel sieht so aus:
\begin{lstlisting}[language=C++, style=CPP]
static PyMethodDef HilfsFunktionen[] =
{
  {"addition_py", add_hilfsfkt, METH_VARARGS, "Addiere zwei Ganzzahlen"},
  {NULL, NULL, 0, NULL}
};
\end{lstlisting}

Nun können wir unsere Modulinitialisierung anfertigen.
Die sieht für ein einfaches Projekt immer so aus (wobei man \lstinline[language=C++,style=CPPinline]|"hilfsmodul"| durch einen beliebigen Namen für sein Modul ersetzten kann und \lcpp{HilfsFunktionen} der Name der Method Table ist).
\begin{lstlisting}[language=C++, style=CPP]
PyMODINIT_FUNC inithilfsmodul(void)
{
  (void) Py_InitModule("hilfsmodul", HilfsFunktionen);
}
\end{lstlisting}

Die gesamte \C-Datei sieht demnach so aus:
\lstinputlisting[language=C++, style=CPP]{code/addition.c}

\subsubsection{Das Modul kompileren und benutzen}
Jetzt ist es an der Zeit, das Modul zu bauen.
Dazu benutzen wir ein einfaches \Python Skript.
Es besteht aus wenigen Zeilen Code.
Wir binden \lpy{distutils.core} ein,
definieren unsere Erweiterung (bestehend aus einem Namen und einer Liste von zu kompilierenden \C-Dateien)
und rufen die Funktion auf, die unsere Erweiterung baut.
Die Namen der Variablen sind selbsterklärend.
\lstinputlisting[language=Python]{code/modul_bauen.py}
Nun rufen wir das Skript zweimal auf.
Im ersten Schritt wird unser Modul gebaut und im zweiten Schritt wird es installiert, so dass es vom \Python-Interpreter CPython genutzen werden kann.
\begin{lstlisting}[language=bash]
python modul_bauen.py build
python modul_bauen.py install --user
\end{lstlisting}
Von jetzt an kann man unsere Funktion durch das Modul \lpy{hilfsmodul} aufrufen.
\begin{lstlisting}
import hilfsmodul
print ('4 + 5 =', hilfsmodul.addition_py(4,5))
\end{lstlisting}

Wird die \C-Datei verändert erneuert man das installierte Modul wie folgt.
\begin{lstlisting}[language=bash]
python modul_bauen.py build
python modul_bauen.py install --user --force
\end{lstlisting}

\subsection{CTypes}
\label{section:python_mit_c:ctypes}
Um ein \Python Projekt (fast) ohne Aufwand um eine bereits in \C implementierte Funktion zu erweitern, kann man ``CTypes'' verwenden.
Das funktioniert sogar, wenn die Funktion nur als ``shared library'' existiert und der Quellcode nicht vorliegt.

Nehmen wir an, wir brauchen eine \Python-Funktion \lpy{subtraktion}, welche die Differenz zweier Gleitkommazahlen berechnet.
Nehmen wir außerdem an, dass die Funktion bereits in kompilierter Form vorliegt und folgende Signatur hat.
\begin{lstlisting}[language=C++,style=CPP]
double subtraktion(double, double);
\end{lstlisting}
Die zugehörige shared library heißt \lcpp{libminus.so}.

Um \lcpp{libminus.so} nutzen zu können, binden wir zunächst das Paket \lpy{ctypes} ein.
\begin{lstlisting}
import ctypes
\end{lstlisting}
Dann laden wir die Bibliothek \lcpp{libminus.so}.
\begin{lstlisting}
bib = ctypes.cdll.LoadLibrary("./libminus.so")
\end{lstlisting}
Auf Funktionen \lcpp{f} und globalen Variablen \lcpp{x} von \lcpp{libminus.so} können wir nun durch \lpy{bib.f} und \lpy{bib.x} zugreifen.
Wir erstellen zunächst eine Variable \lpy{subtraktion_py} die auf die \C-Funktion \lcpp{subtraktion} zeigt.
\begin{lstlisting}
subtraktion_py = bib.subtraktion
\end{lstlisting}
Nun legen wir fest, wie der Rückgabetyp der \C-Funktion in \Python interpretieren werden soll und wie die \Python-Argumente in \C interpretiert werden sollen.
Ist ein \C-Parameter oder der \C-Rückgabewert vom Typ \lcpp{typ} und ist dieser Typ durch den \C-Standard definiert, so können wir die Konvertierung in diesen Typ mit \lpy{ctypes.c_typ} festlegen.
Die Konvertierung in den \C-Typ \lcpp{float} wird also durch \lpy{ctypes.c_float} festgelegt.
Den Rückgabetyp einer Funktion \lpy{f} setzt man mit \lpy{f.restype = ctypes.c_...}.
Die Argumenttypen einer Funktion \lpy{f} wird durch eine Listen von \lpy{ctypes.c_...} angegben, also \lpy{f.argtypes = [ctypes.c_..., ...]}.
Den Rückgabetyp und die Argumente unserer Funktion \lpy{subtraktion_py} setzen wir also so:
\begin{lstlisting}
subtraktion_py.restype  = ctypes.c_double
subtraktion_py.argtypes = [ctypes.c_double, ctypes.c_double]
\end{lstlisting}
Jetzt können wir die \C-Funktion aus \Python heraus ausrufen.
\begin{lstlisting}
print( subtraktion_py(3.5, 6.2) )
\end{lstlisting}

Weiterführende Konzepte wie zum Beispiel variable Argumente findet man in \cite[Library, Generic Operating System Services, CTypes]{Python3}.
