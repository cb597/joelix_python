\subsection{Module einbinden}
\label{section:module:module_einbinden}
\Python ist so mächtig und empfehlenswert, weil es eine riesige Auswahl an guten Modulen gibt, die nur noch auf ihre Nutzung warten.
Module werden mit dem Befehl \lpy{import} eingebunden.
\begin{lstlisting}
import modulname
\end{lstlisting}
Beim Einbinden eines Modul \lpy{modul} wird \lpy{modul} Zeile für Zeile gelesen, geprüft und ausgeführt (also genauso wie beim main-Modul).
Wird ein Modul eingebunden, bekommt es Attribute, zum Beispiel wird der Name des Moduls in der Variable \lpy{__name__} gespeichert.
Um das besser zu verstehen, schauen wir uns folgendes Beispiel an.
Beginnen wir mit der Datei \lpy{hilfsmodul.py}, die so aussehen könnte.
\begin{lstlisting}
if __name__ == '__main__': # Pruefe, ob das Modul das main-Modul ist
  print('Ich bin das main-Modul.')
else:
  print('Ich bin als Modul mit Namen "' + __name__ + '" geladen worden.')
\end{lstlisting}
\PythonDrei interpretiert \lpy{hilfsmodul.py} dann so:
\begin{lstlisting}[language=bash]
python3 hilfsmodul.py
  # Ich bin das main-Modul.
\end{lstlisting}
Nun erstellen wir (im selben Ordner) noch ein \Python Skript, mit Namen \lpy{hauptprogramm.py}, das so aussieht.
\begin{lstlisting}
import hilfsmodul

if __name__ == '__main__': # Pruefe, ob das Modul das main-Modul ist
  print('Ich bin hier der Boss.')
\end{lstlisting}
Schauen wir uns jetzt an, wie \Python dieses Skript interpretiert.
Als erstes wird das Modul \lpy{hilfsmodul} geladen, welches bei uns durch die Datei \lpy{hilfsmodul.py} repräsentiert wird.
Beim Laden des Moduls \lpy{hilfsmodul} wird dieses Zeile für Zeile gelesen, geprüft und ausgeführt.
Da es nicht das main-Modul ist, trägt es einen anderen Namen und dieser wird dann auch ausgedruckt.
Nachdem das Modul \lpy{hilfsmodul} fertig eingebunden wurde, geht es weiter mit dem Skript \lpy{hauptprogramm.py}.
Dieses ist das main-Modul, also wird noch \lpy{'Ich bin hier der Boss.'} ausgedruckt.
Der gesamte Output sieht also so aus:
\begin{lstlisting}[language=bash]
python3 hauptprogramm.py 
  # Ich bin als Modul mit Namen "hilfsmodul" geladen worden.
  # Ich bin hier der Boss.
\end{lstlisting}

Haben wir ein Modul \lpy{hilfsmodul} eingebunden, können wir auf die darin definierten Klassen, Funktionen und Variablen mit dem Präfix \lpy{hilfsmodul.} zugreifen.
Wurde in \lpy{hilfsmodul} beispielsweise eine Funktion \lpy{funktion_eins} definiert, greifen wir auf diese im main-Modul mit \lpy{hilfsmodul.funktion_eins} zu.
Hier ein vollständiges Beispiel.
\begin{lstlisting}
import math             # Importiere das Modul math

s = math.sin(3.141/5.0) # Berechne den Sinus von 3.141/5.0
print(s)                # Drucke den Sinus von 3.141/5.0 aus.
\end{lstlisting}

Es kommt manchmal vor, dass man auf den Inhalts eines Moduls nicht über seinen gegebenen Namen sondern einen frei gewählten Namen zugreifen möchte.
Das kommt zum Beispiel vor, wenn es zwei Versionen einer Bibliothek benutzen möchte (eine ist optimiert, die andere produziert viele Debuginformationen).
In diesem Fall will man nur ganz zu Anfang des Programs festlegen welche Bibliothek eingebunden werden soll ohne weitere Zeilen im Code zu ändern.
In \Python funktioniert das so:
\begin{lstlisting}
import modulname as neuer_modulname
\end{lstlisting}
Nun kann man auf die Klassen, Funktionen und so weiter von \lpy{modulname} mit dem Präfix \lpy{neuer_modulname.} zugreifen.
Schauen wir uns das ganze mal in einem Beispiel an.
In \Python gibt es das Modul \lpy{math} und das Modul \lpy{cmath}.
Das erste bietet die mathematischen Größen und Funktionen an, die im \C-Standard definiert sind;
das zweite bietet analoge mathematische Funktionen an, die auch für komplexe Zahlen definiert sind.
\begin{lstlisting}
import random # Binde random ein
nimm_cmath = random.choice( (True, False) ) # Waehlt True oder False

if nimm_cmath:  # Pruefe, ob das wir cmath importieren wollen
  import cmath as m # Binde cmath ein und nenne es m.
else:
  import math as m  # Binde  math ein und nenne es m.

print( m.sqrt(-1) ) # Druckt 1j wenn cmath eingebunden wurde
                    # und bricht sonst mit einer Ausnahme ab.
\end{lstlisting}

Manchmal will oder muss man eine Hand voll Klassen, Objekte oder Funktionen aus einem Modul \lpy{hilfsmodul} in sein Programm einbinden und gleichzeitig auf das Präfix \lpy{hilfsmodul.} verzichten.
Oft passiert das beim erstellen eines eigenen Moduls (was wir im nächsten Abschnitt behandeln).
Um also eine endliche Menge von Objekten \lpy{obj_1, ..., obj_k} aus einem Modul \lpy{hilfsmodul} direkt einzubinden, nutzen wir den folgenden Code.
\begin{lstlisting}
from hilfsmodul import obj_1, ..., obj_k
\end{lstlisting}
Ähnlich wie eingebundene Module, kann man auch eingebundene Objekte mit \lpy{as} umbennen:
\begin{lstlisting}
from hilfsmodul import obj_1 as neuer_name_1, ..., obj_k as neuer_name_k
\end{lstlisting}
Schauen wir uns das an einem simplen Beispiel an:
\begin{lstlisting}
from math import sin as Sinus, cos as Cosinus, pi as PI

x = Sinus(PI/3.0)
y = Cosinus(PI/3.0)
print( x, y ) # 0.8660254037844386 0.7071067811865476
\end{lstlisting}




