\section{Module}
\label{section:module}
In \Python organisiert man seinen Code (also Klassen, Funktionen, Konstanten, usw.) in sogenannten Modulen.
Ein Modul ist entweder eine einzelne Datei oder ein Ordner (ggf.\ mit weiteren Unterordnern) gefüllt mit \Python Code.

Bekammt der Interpreter eine Skriptdatei \lpy{hauptprogramm.py}, so wird diese interpretiert.
Dabei wird Zeile für Zeile gelesen, geprüft und ausführt.
Es werden also Klassen und Funktionen definiert, sowie Objekte erstellt und Funktionen aufgerufen.
Das interpretierte Skript, sagen wir \lpy{hauptprogramm.py}, wird bereits als Modul verstanden.
Es ist das sogenannte main-Modul.

\subsection{Module einbinden}
\label{section:module:module_einbinden}
\Python ist so mächtig und empfehlenswert, weil es eine riesige Auswahl an guten Modulen gibt, die nur noch auf ihre Nutzung warten.
Module werden mit dem Befehl \lpy{import} eingebunden.
\begin{lstlisting}
import modulname
\end{lstlisting}
Beim Einbinden eines Modul \lpy{modul} wird \lpy{modul} Zeile für Zeile gelesen, geprüft und ausgeführt (also genauso wie beim main-Modul).
Wird ein Modul eingebunden, bekommt es Attribute, zum Beispiel wird der Name des Moduls in der Variable \lpy{__name__} gespeichert.
Um das besser zu verstehen, schauen wir uns folgendes Beispiel an.
Beginnen wir mit der Datei \lpy{hilfsmodul.py}, die so aussehen könnte.
\begin{lstlisting}
if __name__ == '__main__': # Pruefe, ob das Modul das main-Modul ist
  print('Ich bin das main-Modul.')
else:
  print('Ich bin als Modul mit Namen "' + __name__ + '" geladen worden.')
\end{lstlisting}
\PythonDrei interpretiert \lpy{hilfsmodul.py} dann so:
\begin{lstlisting}[language=bash]
python3 hilfsmodul.py
  # Ich bin das main-Modul.
\end{lstlisting}
Nun erstellen wir (im selben Ordner) noch ein \Python Skript, mit Namen \lpy{hauptprogramm.py}, das so aussieht.
\begin{lstlisting}
import hilfsmodul

if __name__ == '__main__': # Pruefe, ob das Modul das main-Modul ist
  print('Ich bin hier der Boss.')
\end{lstlisting}
Schauen wir uns jetzt an, wie \Python dieses Skript interpretiert.
Als erstes wird das Modul \lpy{hilfsmodul} geladen, welches bei uns durch die Datei \lpy{hilfsmodul.py} repräsentiert wird.
Beim Laden des Moduls \lpy{hilfsmodul} wird dieses Zeile für Zeile gelesen, geprüft und ausgeführt.
Da es nicht das main-Modul ist, trägt es einen anderen Namen und dieser wird dann auch ausgedruckt.
Nachdem das Modul \lpy{hilfsmodul} fertig eingebunden wurde, geht es weiter mit dem Skript \lpy{hauptprogramm.py}.
Dieses ist das main-Modul, also wird noch \lpy{'Ich bin hier der Boss.'} ausgedruckt.
Der gesamte Output sieht also so aus:
\begin{lstlisting}[language=bash]
python3 hauptprogramm.py 
  # Ich bin als Modul mit Namen "hilfsmodul" geladen worden.
  # Ich bin hier der Boss.
\end{lstlisting}

Haben wir ein Modul \lpy{hilfsmodul} eingebunden, können wir auf die darin definierten Klassen, Funktionen und Variablen mit dem Präfix \lpy{hilfsmodul.} zugreifen.
Wurde in \lpy{hilfsmodul} beispielsweise eine Funktion \lpy{funktion_eins} definiert, greifen wir auf diese im main-Modul mit \lpy{hilfsmodul.funktion_eins} zu.
Hier ein vollständiges Beispiel.
\begin{lstlisting}
import math             # Importiere das Modul math

s = math.sin(3.141/5.0) # Berechne den Sinus von 3.141/5.0
print(s)                # Drucke den Sinus von 3.141/5.0 aus.
\end{lstlisting}

Es kommt manchmal vor, dass man auf den Inhalts eines Moduls nicht über seinen gegebenen Namen sondern einen frei gewählten Namen zugreifen möchte.
Das kommt zum Beispiel vor, wenn es zwei Versionen einer Bibliothek benutzen möchte (eine ist optimiert, die andere produziert viele Debuginformationen).
In diesem Fall will man nur ganz zu Anfang des Programs festlegen welche Bibliothek eingebunden werden soll ohne weitere Zeilen im Code zu ändern.
In \Python funktioniert das so:
\begin{lstlisting}
import modulname as neuer_modulname
\end{lstlisting}
Nun kann man auf die Klassen, Funktionen und so weiter von \lpy{modulname} mit dem Präfix \lpy{neuer_modulname.} zugreifen.
Schauen wir uns das ganze mal in einem Beispiel an.
In \Python gibt es das Modul \lpy{math} und das Modul \lpy{cmath}.
Das erste bietet die mathematischen Größen und Funktionen an, die im \C-Standard definiert sind;
das zweite bietet analoge mathematische Funktionen an, die auch für komplexe Zahlen definiert sind.
\begin{lstlisting}
import random # Binde random ein
nimm_cmath = random.choice( (True, False) ) # Waehlt True oder False

if nimm_cmath:  # Pruefe, ob das wir cmath importieren wollen
  import cmath as m # Binde cmath ein und nenne es m.
else:
  import math as m  # Binde  math ein und nenne es m.

print( m.sqrt(-1) ) # Druckt 1j wenn cmath eingebunden wurde
                    # und bricht sonst mit einer Ausnahme ab.
\end{lstlisting}

Manchmal will oder muss man eine Hand voll Klassen, Objekte oder Funktionen aus einem Modul \lpy{hilfsmodul} in sein Programm einbinden und gleichzeitig auf das Präfix \lpy{hilfsmodul.} verzichten.
Oft passiert das beim erstellen eines eigenen Moduls (was wir im nächsten Abschnitt behandeln).
Um also eine endliche Menge von Objekten \lpy{obj_1, ..., obj_k} aus einem Modul \lpy{hilfsmodul} direkt einzubinden, nutzen wir den folgenden Code.
\begin{lstlisting}
from hilfsmodul import obj_1, ..., obj_k
\end{lstlisting}
Ähnlich wie eingebundene Module, kann man auch eingebundene Objekte mit \lpy{as} umbennen:
\begin{lstlisting}
from hilfsmodul import obj_1 as neuer_name_1, ..., obj_k as neuer_name_k
\end{lstlisting}
Schauen wir uns das an einem simplen Beispiel an:
\begin{lstlisting}
from math import sin as Sinus, cos as Cosinus, pi as PI

x = Sinus(PI/3.0)
y = Cosinus(PI/3.0)
print( x, y ) # 0.8660254037844386 0.7071067811865476
\end{lstlisting}





\subsection{Eigene Module in Python bereitstellen}
\label{section:module:module_selbst_schreiben}
Um in \Python Bibliotheken zur Verfügung zu stellen, packt man (ähnlich wie in \CC) seinen gesamten Code in ein Modul.
Das kann zwar ein einziges \Python-Skript sein, aber so gut wie immer macht es viel mehr Sinn, seinen Code in mehreren Dateien (die auf mehrere Ordnern verteilt sind) zu organisieren.
Es ist gängige Praxis seine logischen Einheiten in einem gewurzelten Baum zu organisieren.
Jedes Blatt entspricht einer Datei und alle anderen Knoten sind Ordner, welche die darunter liegenden Knoten (also Ordner und Dateien) enthalten.
Außerdem ist es oft klug einen gewissen Anteil des Codes dem Benutzer nicht direkt zugänglich machen.
So kann man als Anbieter der Bibliothek kleine Details ändern, ohne dass sich der Benutzer der Bibliothek an neue Funktionen gewöhnen muss.

Wir besprechen anhand eines Beispiel die einfachsten Möglichkeiten ein eigenes Modul zu erstellen.
Es sei hier noch gesagt, dass Module auch Untermodule bereitstellen können.


\subsubsection{Skripte als Module einbinden}
Im aller einfachsten Fall ist ein Modul nichts anderes als eine \Python-Skript.
Das haben wir im vergangenen Abschnitt bereits gesehen.
Nehmen wir an, wir haben bereits eine \Python-Skript \lpy{prog.py} in einem Order \lpy{wurzel} erstellt und mit sinnvollem Code gefüllt.
Wenn wir nun ein weiteres \Python-Skript \lpy{hauptprog.py} im selben Ordner (also \lpy{wurzel}) anlegen, können wir \lpy{prog.py} mithilfe von \lpy{import} laden.
Beim Importieren wird die Dateiendung weggelassen:
\begin{lstlisting}
# Datei: hauptprog.py
import prog # Importiert prog.py
...
\end{lstlisting}
Beim importieren des \Python-Skripts \lpy{prog.py} wird dieses Zeile für Zeile gelesen, geprüft und ausgeführt.
Danach kann man auf die Funktionen, Klasses und Variablen mithilfe des Präfixes \lpy{prog.} zugreifen.
Also greifen wir auf die Variable \lpy{x} die in \lpy{prog.py} definiert wurde mit \lpy{prog.x} zu.

\subsubsection{Ordner als Module einbinden}
Im nächst einfachen Fall ist ein Modul nichts anderes als ein Order gefüllt mit \Python-Skripten.
Gehen wir davon aus, dass wir einen Ordner \lpy{skript_bib} haben, in dem die \Python-Skripte \lpy{skript_eins.py}, \lpy{skript_zwei.py} und \lpy{skript_drei.py} liegen haben.
Damit der \Python-Interpreter den Ordner \lpy{skript_bib} als Modul interpretieren kann, muss der Ordner ein \Python-Skript mit dem Namen \lpy{__init__.py} enthalten.
Dieses Skript wird beim Einbinden des Moduls Zeile für Zeile gelesen, geprüft und ausgeführt.
Alle anderen Skripte werden nicht automatisch interpretiert, denn das ist überlicherweise die Aufgabe von \lpy{__init__.py}.
In \lpy{__init__.py} bindet man dann entweder alle Klassen, Funktionen und Variablen der anderen Skripte ein.
Das geschieht mithilfe von
\begin{lstlisting}
from skript_eins.py import *
\end{lstlisting}
oder man bindet nur die benötigen Klassen, Funktionen und Variablen der anderen Skripte ein.
Das geschieht mithilfe von
\begin{lstlisting}
from skript_zwei.py import klasse_a, klasse_b, funktion_z
\end{lstlisting}
Sobald das Modul \lpy{skript_bib} importiert ist, kann man auf alle Klassen, Funktionen und Variablen, die dem Skript \lpy{__init__.py} bekannt sind, mithilfe des Präfixes \lpy{skript_bib.} zugreifen.

Wir besprechen das nun anhand eines Beispiels.
Angenommen wir wollen ein Modul \lpy{JoelixBlas} anfertigen, dass sich wie die Bibliothek \lcpp{joelixblas} verhält (siehe \cite{joelixC}).
Das heißt, wir wollen ein Modul, das Vektoren und (dünnbesetzte Matrizen) anbietet und elementare Operationen (wie zum Beispiel Matrix-Vektor-Multiplikation) bereitstellt.
Zunächst erstellen wir ein Verzeichnis mit dem Namen unseres Moduls, also \lpy{JoelixBlas}.
Die logischen Einheiten \lpy{JoelixVektor} und \lpy{JoelixMatrix} sind zu gewissen Teilen voneinander unabhängig und wir entscheiden uns dazu, sie in zwei separate Dateien zu lagern.
Unsere Verzeichnisstruktur sollte zu diesem Zeitpunkt so aussehen:
\begin{lstlisting}[language=plain]
JoelixBlas
  +--joelix_vektor.py
  +--joelix_matrix.py
\end{lstlisting}
Nun implementieren wir die Klassen \lpy{JoelixVektor} und \lpy{JoelixMatrix}.
Um \lpy{JoelixVektor} zu implementieren, brauchen wir die Klasse \lpy{JoelixMatrix} nicht zu kennen.
Die Datei \lpy{joelix_vektor.py} sieht so aus:
\begin{lstlisting}
class JoelixVektor:
  """Vektorklasse der JoelixBlas."""
  ...
  def __add__(self, other): # Implementiert Vektor-Vektor-Addition
    """Implementiert Vektor-Vektor-Addition."""
    if type(other) is type(self):
      # Vektor-Vektor-Addition
  ...
\end{lstlisting}
Die Matrix-Matrix-Multiplikation und die Matrix-Vektor-Multiplikation soll durch eine spezielle Memberfunktion von \lpy{JoelixMatrix} realisiert werden.
Da wir die Klasse \lpy{JoelixVektor} für die Matrix-Vektor-Multiplikation benötigen, binden wir \lpy{joelix_vektor} in \lpy{joelix_matrix} ein:
Die Datei \lpy{joelix_matrix.py} sieht so aus:
\begin{lstlisting}
# JoelixMatrix.py
from joelix_vektor import JoelixVektor

class JoelixMatrix:
  """Duennbesetzte Matrix der JoelixBlas."""
  ...
  def __mul__(self, other): # Definiere Matrix-X-Multiplikation
    """Implementiert die Matrix-X-Multiplikation."""
    if type(other) is type(self):
      # Matrix-Matrix-Multiplikation
    elif type(other) is JoelixVektor:
      # Matrix-Vektor-Multiplikation
  ...
\end{lstlisting}
Damit wir den Ordner \lpy{JoelixBlas} als Modul einbinden können, braucht der Ordner \lpy{JoelixBlas} noch das \Python-Skript \lpy{__init__.py}.
In diesem importieren wir die Klasse \lpy{JoelixVektor} aus \lpy{joelix_vektor.py} und die Klasse \lpy{JoelixMatrix} aus der Datei \lpy{joelix_matrix.py}.
\begin{lstlisting}
# Datei: __init__.py
from joelix_vektor import JoelixVektor
# Jetzt kennt __init__.py die Klasse JoelixVektor
from joelix_matrix import JoelixMatrix
# Jetzt kennt __init__.py die Klasse JoelixMatrix
\end{lstlisting}
Unsere Verzeichnisstruktur sollte also so aussehen:
\begin{lstlisting}[language=plain]
JoelixBlas
  +--__init__.py
  +--joelix_vektor.py
  +--joelix_matrix.py
\end{lstlisting}

Jetzt können wir das Modul \lpy{JoelixBlax} in einem \Python-Programm verwenden.
Zum Beispiel erstellen wir ein \Python-Skript mit dem Namen \lpy{test.py} das imselben Order liegt, wie der Ordner \lpy{JoelixBlas}.
Unsere Verzeichnisstruktur sollte also so aussehen:
\begin{lstlisting}[language=plain]
mein_programm
  +--test.py
  +--JoelixBlas
       +--__init__.py
       +--joelix_vektor.py
       +--joelix_matrix.py
\end{lstlisting}
Das \Python-Skript \lpy{test.py} sieht so aus:
\begin{lstlisting}
import JoelixBlas

def main():
  Vek = JoelixBlas.JoelixVektor(dim=3)
  Vek[0] = 5.0
  Vek[1] = 4.6
  Vek[2] = 1.8
  print( Vek )

if __name__ == '__main__':
  main()
\end{lstlisting}

\subsection{Empfohlene Module}
\label{section:module:empfohlene_module}
In diesem Abschnitt beschreiben wir eine Hand voll Module, die wir für besonders wichtig halten.
Um dieses Skript nicht unnötig lang werden zu lassen, besprechen wir die Module nur ganz oberflächlich und verweisen die interessierte Leserin auf die offizielle Dokumentation der hier besprochenen Module.
Die Module in den Abschnitten \ref{section:module:empfohlene_module:re},
\ref{section:module:empfohlene_module:copy},
\ref{section:module:empfohlene_module:math},
\ref{section:module:empfohlene_module:cmath},
\ref{section:module:empfohlene_module:random},
\ref{section:module:empfohlene_module:itertools},
\ref{section:module:empfohlene_module:pickle},
\ref{section:module:empfohlene_module:os},
\ref{section:module:empfohlene_module:time},
\ref{section:module:empfohlene_module:argparse},
\ref{section:module:empfohlene_module:multiprocessing},
\ref{section:module:empfohlene_module:subprocess},
und \ref{section:module:empfohlene_module:sys}
sind (in dieser Reihenfolge) durch den \Python-Standard beschrieben.
Im Abschnitt \ref{section:module:empfohlene_module:rest} empfehlen wir noch einige Module, die für Mathematiker interessant sein dürften.


\subsubsection{re}
\label{section:module:empfohlene_module:re}
Ein ``regulärer Ausdruck'' ist ein String (mit vielen Sonderzeichen) mitdem eine Menge von gültigen Strings beschrieben wird.
Zum Beispiel ist die Menge der gültigen String zum reguläre Ausdruck \lpy{*.py}, die Menge der Strings die mit \lpy{.py} enden und einen beliebigen (eventuell leeren) Wortanfang besitzen.
Eigentlich verdienen reguläre Ausdrücke ein eigenes Kapitel, das würde dieses Skript aber inhaltlich sprengen.
Wir machen hier nur darauf aufmerksam, dass das Modul \lpy{re} viele Methoden zum Arbeiten mit regulären Ausdrücken bereitstellt.
Wenn man mit regulären Ausdrücken vertraut ist, ließt man in \cite[Library, Text Processing Services, Regular expression operations]{Python3} wie das Modul \lpy{re} zu benutzen ist.
Sonst verweisen wir den Leser auf \cite[Kapitel 33]{LottPython}.


\subsubsection{copy}
\label{section:module:empfohlene_module:copy}
Wir haben in Abschnitt \ref{section:std_data_types:sequenzen:kopien} gesehen, dass Kopien von Sequenzen immer oberflächeliche Kopien sind.
Das ist zwar schnell, kann aber zu ungewollten Effekten führen wenn man nicht vorsichtig ist.
Um echte Kopien zu erstellen, benutzt man das Modul \lpy{copy}.
Die darin implementierte Funktion \lpy{copy.deepcopy} erstellt sogenannte tiefe Kopien.


\subsubsection{math}
\label{section:module:empfohlene_module:math}
Das Modul \lpy{math} stellt die mathematischen Funktionen und Konstanten zur Verfügung, die der \C-Standard beschreibt.
Zum Beispiel zieht man die Wurzel mit \lpy{math.sqrt}.
\begin{lstlisting}
import math
math.sqrt(4.0)
\end{lstlisting}


\subsubsection{cmath}
\label{section:module:empfohlene_module:cmath}
Das Modul \lpy{cmath} beschreibt dieselben mathematischen Funktionen und Konstanten, allerdings ist der Definitions- und Wertebereich \lpy{complex} statt \lpy{float}.
Zum Beispiel zieht man die komplexe Wurzel mit \lpy{cmath.sqrt}.
\begin{lstlisting}
import cmath
cmath.sqrt(-1.0)
\end{lstlisting}


\subsubsection{random}
\label{section:module:empfohlene_module:random}
Das Modul \lpy{random} implementiert einfache Pseudozufallszahlengeneratoren für verschiedene Verteilungen.
Um aus einer Sequenz ein Element (bezüglich der Gleichverteilung) zufällig auszuwählen, nutzt man die Funktion \lpy{random.choice}.
Beispielsweise kann man einen Buchstaben aus dem String \lstinline[style=PyInline]|"Hallo Welt"| wie folgt zufällig auswählen:
\begin{lstlisting}
import random
random.choice( "Hallo Welt" )
\end{lstlisting}


\subsubsection{itertools}
\label{section:module:empfohlene_module:itertools}
Um aus Sequenzen (oder allgemeiner ``iterierbaren Objekten``) neue iterierbare Objekte zu erstellen, benutzt man das Modul \lpy{itertools}.
Dieses Modul ist nicht nur speicher- und laufzeiteffizient sondern auch sehr mächtig, wenn man es richtig einzusetzen weiß.
An dieser Stelle sei gesagt, dass die Funktionen von \lpy{itertools} bei gegebenem Input keine Sequenzen erstellen (die dann anschließend Element für Element ausgelesen werden können),
sondern sie erstellen sogenannte ''iterierbares Objekte``, die den aktuell zu verarbeiteten Eintrag zur Laufzeit generieren.
Dass spart bei großem Input wertvollen Speicher und ist meistens schneller.
Wir stellen hier nur zwei sehr einfache Funktionen vor und legen der interessierten Leserin \cite[Library, Funktional Programming Tools, itertools]{Python3} an Herz.

Um aus zwei Sequenzen \lpy{A} und \lpy{B} eine Sequenz \lpy{((x,y) for x in A for y in B)} zu erstellen, reicht der angegebene Ausdruck,
allerdings ist die Funktion \lpy{itertools.product} schneller und speichereffizienter.
Hier ein Beispiel:
\begin{lstlisting}
import itertools
A = 'Hallo Welt'
B = ( 1.0, 4.5, 'sss' )
C = itertools.product(A,B) # Aequivalent zu ((x,y) for x in A for y in B)
\end{lstlisting}
Außerdem kann man \lpy{product} eine beliebige Anzahl von Sequenzen übergeben.

Um (eine Sequenz) alle(r) Permutationen einer gegebenen Sequenz \lpy{s} zu erhalten, nimmt man die Funktion \lpy{itertools.permutations}.
Beispielsweise erhält man alle Permutationen des String \lpy{'Valhalla'} wie folgt:
\begin{lstlisting}
import itertools
p = itertools.permutations('Valhalla')
for x in p:
  print(x)  # Druckt:
            # ('V', 'a', 'l', 'h', 'a', 'l', 'l', 'a')
            # ('V', 'a', 'l', 'h', 'a', 'l', 'a', 'l')
            # ('V', 'a', 'l', 'h', 'a', 'l', 'l', 'a')
            # ...
\end{lstlisting}
Will man (eine Sequenz) alle(r) Permutationen von \lpy{s} der Länge \lpy{r}, haben, übergibt man \lpy{permutations} noch den Parameter \lpy{r}:
\begin{lstlisting}
import itertools
p = itertools.permutations('Valhalla', 2)
for x in p:
  print(x)  # Druckt:
            # ('V', 'a')
            # ('V', 'l')
            # ('V', 'h')
            # ...
\end{lstlisting}
In beiden Fällen ist es wichtig zu wissen, dass man die gesamte Symmetrische Gruppe auf (einem variierenten Teil der Länge $r$) der Sequenz operieren lässt.
Insbesondere besteht \lpy{itertools.permutations(s)} aus $l!$ vielen Elementen, wenn \lpy{s} eine Sequenz der Länge $l$ ist, selbst wenn manche Werte in \lpy{s} mehrfach vorkommen.


\subsubsection{pickle}
\label{section:module:empfohlene_module:pickle}
Um Objekte in \Python auf der Festplatte zu speichern und von der Festplatte zu laden, kann man das Modul \lpy{pickle} benutzen.
Das Speichern und Laden ist plattformunabhängig, solange die gleiche \Python-Version verwendet wird, das heißt,
man kann mit \lpy{pickle} gespeicherte Daten zwischen den Betriebssystemen GNU/Linux, Mac und Windows hin und hertauschen, solange man dieselbe \Python-Version verwendet.

Um mehrere Objekte in einer Datei zu speichern, muss zuerst das Modul \lpy{pickle} importiert werden.
Nun muss man eine Datei \lpy{datei} zum Schreiben im ''Bytemodus`` geöffnet werden.
Dies geschieht durch den Ausdruck \lpy{datei = open('pfad', 'wb')}.
Im Anschluss kann man mit \lpy{pickle.dump(obj, datei)} ein Objekt nach dem anderen in die zuvor geöffnete Datei \lpy{datei} schreiben.
Im folgenden Beispiel speichern wir die Zahlen \lpy{0, ..., 9}:
\begin{lstlisting}
import pickle                # Importiert pickle
datei = open( 'test', 'wb' ) # Oeffnet Datei 'test' im Bytemodus
for i in range(10):          # Fuer 0 <= i < 10:
  pickle.dump(i, datei)      # Schreibe i in datei
datei.close()                # Datei schliessen
\end{lstlisting}

Um mehrere Objekte aus einer Datei zu lesen, muss ebenfalls zuerst das Modul \lpy{pickle} importiert werden.
Nun muss man eine Datei \lpy{datei} zum Lesen im ''Bytemodus`` geöffnet werden.
Dies geschieht durch den Ausdruck \lpy{datei = open('pfad', 'rb')}.
Im Anschluss kann man mit \lpy{pickle.load(datei)} ein Objekt nach dem anderen aus der zuvor geöffneten Datei \lpy{datei} lesen.
Das Lesen der zuvor gespeicherten Objekte funktioniert nach dem ''FIFO``-Prinzip, d.h.\ Daten die zuerst geschrieben wurde, werden zuerst gelesen.
Der Funktionsaufruf \lpy{pickle.load(datei)} gibt das gespeicherte Objekt zurück oder löst eine Ausnahme vom Typ \lpy{EOFError} aus sofern die Datei keine weiteren Objekte enthält.
Im folgen Beispiel lesen wir alle Daten aus einer Datei aus.
\begin{lstlisting}
import pickle                 # Importiert pickle
datei = open( 'test', 'rb' )  # Oeffnet Datei 'test' im Bytemodus
dateiende_erreicht = False    # Soll True sein wenn Datei am Ende ist
objekte = []                  # Liste der gelesenen Objekte
while dateiende_erreicht == False: # Solange die Datei nicht am Ende ist
   try:                       # Versuche
     obj = pickle.load(datei) # Ein Objekt zu lesen
     objekte.append(obj)      # Das gelesene Objekt zur Liste hinzufuegen
   except EOFError:           # Fange EOFError ab (= Dateiende erreicht)
     dateiende_erreicht = True # Setze dateiende_erreicht auf True
datei.close()                 # Datei schliessen
print(objekte)                # Drucke Liste der Objekte aus
\end{lstlisting}


\subsubsection{os}
\label{section:module:empfohlene_module:os}
Das Modul \lpy{os} ermöglicht es uns betriebssystemespeziefische Aufgaben plattformunabhängig zu lösen.
Beispielsweise müssen Unterordner im Dateipfad unter Windows mit \lstinline[style=PyInline]|'\'| und unter GNU/Linux und Mac mit \lpy{'/'} kennzeichnen.
Als \Python-Programmererin hat man an solchen lästigen Fallunterscheidungen kein Interesse.
Mit \lpy{os.mkdir('pfad')} erstellt man neue Ordner.
Um zum Beispiel zu testen, ob ein Dateipfad einen Ordner oder eine normale Datei benennt, kann man folgende Funktionen verwenden.
\begin{lstlisting}
import os
os.path.isfile('pfad') # True gdw 'pfad' eine normale Datei ist
os.path.isdir('pfad')  # True gdw 'pfad' ein Ordner ist
\end{lstlisting}


\subsubsection{time}
\label{section:module:empfohlene_module:time}
Wie der Name des Moduls \lpy{time} vermuten lässt, benutzt man es um die Zeit zu messen oder eine Pause einzulegen.
Um einen Prozess für \lpy{k} Sekunden schlafen zu legen, nutzt man die Funktion \lpy{time.sleep(k)}.
Um die Echtzeit zu messen, benutzt man die Funktion \lpy{time.perf_counter}.
Die CPU-Zeit eines Prozesses misst man mit \lpy{time.process_time}.
Hier ein Beispiel:
\begin{lstlisting}
import time
echtzeit_start = time.perf_counter()
cpuzeit_start  = time.process_time()
# Hier eine aufwaendige Berechnung einsetzen, welche die CPU zu
# 100% auslastet und ca. 1000 Sekunden dauert
time.sleep(500) # Warte nochmal 500 Sekunden (CPU-Kosten ~ 0%)
echtzeit_dauer = time.perf_counter() - echtzeit_start # ~ 1500
cpuzeit_dauer  = time.process_time() - cpuzeit_start  # ~ 1000
\end{lstlisting}


\subsubsection{argparse}
\label{section:module:empfohlene_module:argparse}
Mit dem Modul \lpy{argparse} kann man ganz einfach benutzerfreundliche Kommandozeileninterfaces herstellen.
Kurz gesagt, legt man mit \lpy{argparse} fest, welche Kommandozeilenparameter es gibt, welche Parameter optional sind und welche notwendig sind.
Sind die Kommandozeilenparameter gültig, wird ein Verzeichniss der Parameter erstellt, die den Programm dann zur Verfügung stehen.

Zuerst erstellen wir einen ''Argumentparser`` mit
\begin{lstlisting}
import argparse
parser = argparse.ArgumentParser(description='Beschreibung d. Programms')
\end{lstlisting}
Als Parameter \lpy{description} übergibt man einen String, der das Programm genau beschreibt.
Als nächstes erklären wir, welche Parameter dem Programm per Kommandozeile übergeben werden können.
Außerdem können wir Standardwerte festlegen oder bestimmen, welche Parameter optional sind und welche nicht.
Dazu benutzen wir die Memberfunktion \lpy{add_argument}.
Neben dem Namen der Kommandozeilenoption, erklären wir noch einige optionale Parameter der Funktion.
\begin{lstlisting}
name     # Name des Parameters z.B. spam
action   # Wir besprechen hier:
         # (1) 'store': der nach spam folgende Parameter wird gespeichert
         # (2) 'store_true': speichert True (wenn spam ein Parameter ist)
type     # Legt den Typ fest, der gespeichert wird
default  # Legt einen Standardwert fest, auch wenn spam kein Parameter ist
required # Wenn required=True ist, muss spam Parameter sein
help     # String zur Erklaerung des Parameter fuer den Benutzer
\end{lstlisting}
Um Übergebene Kommandozeilenparameter auf Gültigkeit zu prüfen und im Anschluss Objekt mit den entsprechenden Parametern zu erstellen, rufen wir die Memberfunktion \lpy{parse_args} auf.
Diese gibt (auch wenn wir in diesem Skript nicht genug auf Namespaces eingegangen sind) einen Namespace, der die übergebenen Parameter beschreibt.
Aus diesem machen wir ein Verzeichniss mit der eingebauten Funktion \lpy{vars}.
Die Schlüssel des Verzeichnisses sind die Namen der Parameter:
\begin{lstlisting}
argumente = vars(parser.parse_args()) # Erstellt Verzeichniss d. Argumente
print(argumente)                      # Druckt das Verzeichniss aus
\end{lstlisting}
Das ganze versteht man am besten anhand eines Beispiels:
Hier wollen wir ein simples Programm schreiben, das zwei Zahlen addiert oder subtrahiert.
Dabei sollen dem Programm die Zahlen \lpy{x} und \lpy{y} und der Operand \lpy{op} in beliebiger Reihenfolge übergeben werden.
Wenn man einen Parameter vergisst soll die Hilfe ausgegeben werden, die beschreibt wie das Programm funktioniert.
Die Berechnungsfunktion sieht so aus:
\begin{lstlisting}
def berechne(x,y,op):
  if op == 'plus':
    print('{} + {} = {}'.format(x,y,x+y))
  else:
    print('{} - {} = {}'.format(x,y,x-y))
\end{lstlisting}
Jetzt erstellen wir den Argumentenparser:
\begin{lstlisting}
import argparse
parser = argparse.ArgumentParser(description='Berechnet x op y')
parser.add_argument(
  '-x', action='store', type=float, required=True, help='Die Zahl x')
parser.add_argument(
  '-y', action='store', type=float, required=True, help='Die Zahl y')
parser.add_argument(
  '-op', action='store', type=str, required=True, help='plus oder minus')
\end{lstlisting}
Jetzt setzen wir alles im main-modul zusammen.
Das heißt, wir lassen der Argumentparser die übergebenen Parameter interpretieren und rufen dann die Funktion \lpy{berechne} auf.
Der vollständige Code ist wie folgt.
\begin{lstlisting}
import argparse
parser = argparse.ArgumentParser(description='Berechnet x op y')
parser.add_argument(
  '-x', action='store', type=float, required=True, help='Die Zahl x')
parser.add_argument(
  '-y', action='store', type=float, required=True, help='Die Zahl y')
parser.add_argument(
  '-op', action='store', type=str, required=True, help='plus oder minus')

def berechne(x,y,op):
  if op == 'plus':
    print('{} + {} = {}'.format(x,y,x+y))
  else:
    print('{} - {} = {}'.format(x,y,x-y))

def main():
  args = vars(parser.parse_args()) # Erstellt Verzeichniss d. Argumente
  print(args)
  berechne(**args)

if __name__ == '__main__':
  main()
\end{lstlisting}


\subsubsection{multiprocessing}
\label{section:module:empfohlene_module:multiprocessing}
Auch in \Python können wir nebenläufige Programme schreiben, indem wir (mehr oder weniger) voneinander unabhängige Programmabschnitte parallel ausführen.
Im aller einfachsten Fall, wo wir eine Funktionen \lpy{f} mit Parametern \lpy{p_1, ..., p_k} parallel aufrufen möchten, gehen wir wie folgt vor:
Wir importieren \lpy{multiprocessing} und erstellen einen sogenannten ''Arbeiterpool``.
Ein Arbeiterpool ist eine Menge von Arbeiterprozessen, die ihre Aufgaben parallel bearbeiten können.
Man kann die Größe des Pools mit dem Parameter \lpy{processes} selbst angeben oder das Betriebssystem die maximale Anzahl von Arbeitern selbst bestimmen lassen.
Unser Programmcode sieht bis jetzt zum Beispiel so aus:
\begin{lstlisting}
import multiprocessing
if __name__ == '__main__':
  arbeiterpool = multiprocessing.Pool(processes=2) # Pool der Groesse 2
\end{lstlisting}
Haben wir eine Funktion \lpy{f} definiert und eine Liste \lpy{data} von (Listen von) Argumenten, die wir \lpy{f} übergeben,
so können wir diese Funktionen wie folgt parallel ausführen.
\begin{lstlisting}
import multiprocessing
import time
def f( x,t ):
  time.sleep(t)
  print(x)
if __name__ == '__main__':
  arbeiterpool = multiprocessing.Pool(processes=2) # Pool der Groesse 2
  data = [('Hallo', 2), ('Welt',1), ('Python',3), ('ist',1), ('cool',2)]
  arbeiterpool.starmap( f, data )
\end{lstlisting}
Es ist ganz wichtig, dass wir hier abfragen, ob das Modul das main-Modul ist,
denn beim Parallelisieren mit \lpy{multiprocessing} wird das main-Modul ggf.\ in sich selbst neu eingebunden.
Wenn man mit Funktionen arbeitet, die nur ein Argument bekommt, nimmt man oft \lpy{arbeitspool.map} statt \lpy{arbeitspool.starmap}:
\begin{lstlisting}
import multiprocessing
def f( x ):
  print(x)
if __name__ == '__main__':
  arbeiterpool = multiprocessing.Pool(processes=2) # Pool der Groesse 2
  data = ['Hallo', 'Welt', 'Python', 'ist', 'cool']
  arbeiterpool.map( f, data )
\end{lstlisting}


\subsubsection{subprocess}
\label{section:module:empfohlene_module:subprocess}
Manchmal möchte oder muss man ein anderes Programm aus \Python heraus aufrufen.
Dazu verwendet man das Modul \lpy{subprocess}.
Man kann mit dem aufgerufenen Programm kommunizieren (zum Beispiel Eingaben oder Signale Senden und Ausgaben empfangen),
wie das genau funktioniert erklären wir hier jedoch nicht und verweisen auf \cite[Library, Concurrent Execution, Subprocess management]{Python3}.
Ein Programm \lpy{programm} mit einer Kommandozeilen Parametern \lpy{'par_1', ..., 'par_k'} startet man wie folgt:
\begin{lstlisting}
import subprocess
p = subprocess.run([programm, par_1, ..., par_k], stdout=subprocess.PIPE)
\end{lstlisting}
Im Anschluss liegt der Rückgabewert des Programms \lpy{programm} in \lpy{p.returncode} und der Standardoutput von \lpy{programm} liegt in \lpy{p.stdout}.

Wir machen hier darauf aufmerksam, dass \lpy{subprocess.run} erst ab der \Python-Version 3.5 existiert.
Für führere \Python-Versionen, kann man \lpy{subprocess.call} verwenden; diese Funktion gibt aber nur den Rückgabewert des Programms zurück, den Standardoutput muss man sich anders besorgen.


\subsubsection{sys}
\label{section:module:empfohlene_module:sys}
Das Modul \lpy{sys} hält einige Variablen und Funktionen bereit, die start vom \Python-Interpreter abhängen.
Zum Beispiel ist die \Python-Version durch das Tupel \lpy{sys.version_info} beschrieben und der Standardoutput wird durch \lpy{sys.stdout} abgebildet.


\subsubsection{Weitere Python Module}
\label{section:module:empfohlene_module:rest}
Um numerische Berechnungen durchzuführen, gibt es das \Python-Projekt \lpy{scipy}.
Es ist sehr gut und baut auf \lpy{numpy} auf.

Sowohl numerische Berechnungen als auch algebraische Modellierung kann man mit dem \Python-Projekt \lpy{SageMath} durchführen.
Dieses greift auf eine riesige Bibliothek von Projekten zu, die in verschiedenen Programmiersprachen geschrieben sind.
Es ist sehr gut.

Um ein bisschen zu malen oder mit Bildern zu arbeiten empfehlen wir noch \lpy{pygame}, \lpy{cairo} und \lpy{PIL}.
Auch diese Module sind sehr gut.

