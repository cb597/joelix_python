\section{Das Python Datenmodell}
\label{section:datamodel}
Ein wesentlicher Unterschied zwischen \CC und \Python ist das Verhalten von Daten im Speicher und den Zugriff darauf.
Um den Unterschied zu \CC klarer zu machen, beginnen wir mit einer kurzen Wiederholung


\subsection{Wiederholung: Daten in C und C++}
\label{section:datamodel:cc}
In \CC gibt es Variablen.
Diese bestehen aus einer ``Speicheradresse'', einem ``Typ'' und dem dort gespeicherten ``Wert''.
Sie verhalten sich also wie ein zusammenhängendes Stück Speicher, indem ein Wert gespeichert wird und wir wissen, wie wir diesen Wert zu interpretieren haben.
Nehmen wir beispielsweise folgende Codezeile.
\begin{lstlisting}[style=CPP]
int32_t a = 0;
int32_t b = 3;
a = b;
\end{lstlisting}
In der ersten Zeile erstellen wir eine Variable vom Typ \lcpp{int32_t}, d.h.\ wir haben nun ein zusammenhängendes Stück Speicher indem der Wert einer ganzen 32-bit-Zahl gespeichert werden kann,
und legen dort den Wert der Expression \lcpp{0} ab.
In der zweiten Zeile erstellen wir eine weitere Variable vom Typ \lcpp{int32_t} und legen dort den Wert der Expression \lcpp{3} ab.
In der dritten Zeile legen wir den Wert der Expression \lcpp{b} in den zu \lcpp{a} gehörigen Speicher ab.
Nun ist der in \lcpp{a} gespeicherte Wert \lcpp{3} und der in \lcpp{b} gespeicherte Wert ebenfalls \lcpp{3}.
Wenn \lcpp{b} stattdessen vom Typ \lcpp{double *} wäre, führte Zeile 3 zu einem Compilerfehler.


\subsection{Daten in Python}
\label{section:datamodel:python}
Das \Python Datenmodell ist intelektuell nicht so einfach zu fassen wir das Datenmodell von \CC.
Es ist zwar keine Magie, dennoch kann es einige Tage oder gar Wochen dauern kann, bis man sich daran gewöhnt hat.

In \Python gibt es ``Objekte'' und ``Variablen''.
Objekte repräsentieren Daten und Variablen referenzieren auf Objekte.
Zunächst besprechen wir Objekte genauer und kümmer uns im Anschluss um Variablen.


\subsubsection{Objekte}
\label{section:datamodel:python:objekte}
Objekte bestehen aus einer ``Identität'', einem ``Typ'' und einem ``Wert''.
Sie verhalten sich wie ein zusammenhängendes Stück Speicher, in dem ein Wert gespeichert wird,
also ganz genauso wie sich Variablen in \CC verhalten.
Die Identität eines (existierenden) Objekts ändert sich nie und bestimmt ein Objekt (während der Laufzeit) eindeutig.
Wenn man mag, kann man sich die Identität wie die Speicheradresse des Objekts vorstellen (auch wenn man in \Python keine Pointer verwendet).

Ein jedes Objekt ist entweder ``mutable'' und kann nach seiner Erzeugung verändert werden oder es ist ``immutable'' und ist nach seiner Erzeugung konstant.
Dieses Konzept gibt es ebenfalls in \CC:
Alle Typen in \CC sind ohne weiteres ``mutable'', es sei denn, sie bekommen das Präfix \lcpp{const} bei ihrer Definition, denn dann sind sie ``immutable''.

In \Python sind folgende Objekte immutable:
die ganze Zahl \lpy{3} (siehe Abschnitt \ref{section:std_data_types:zahlen}),
das Tupel \lpy{(1, 'a')} (siehe Abschnitt \ref{section:std_data_types:sequenzen}) und
der leere Typ \lpy{None} (siehe Abschnitt \ref{section:std_data_types:none}).
Die folgenden Objekte sind mutable:
Die Liste \lpy{[1, 'a']} (siehe Abschnitt \ref{section:std_data_types:sequenzen}),
das Verzeichnis \lstinline[style=Pyinline]|{ 'Felix' : 405, 'Jonathan' : 599  }| (siehe Abschnitt \ref{section:std_data_types:verzeichnisse})
und die meisten Klassen (siehe Abschnitt \ref{section:klassen}).


\subsubsection{Variablen}
\label{section:datamodel:python:variablen}
In \Python haben Variablen keinen Typ.
Wir wiederholen nochmal: In \Python haben Variablen keinen Typ.
Und ja, in der Tat, wir wiederholen nochmal: In \Python haben Variablen keinen Typ.
Variablen referenzieren auf Objekte; sie sind sozusagen die (temporären) Namen von Objekten.
Wenn man mag, kann man sich Variablen wie \lcpp{(void)}-Pointer vorstellen (auch wenn man in \Python keine Pointer verwendet).

Variablen beziehen sich immer auf ein Objekt.
Beim Erstellen einer Variable muss ihr konsequenterweise direkt ein Objekt zugewiesen werden.
Betrachten wir ein einfaches Beispiel.
\begin{lstlisting}
a = 'Hallo' # Erstelle a; a referenziert das Objekt 'Hallo'.
b = 'Welt'  # Erstelle b; b referenziert das Objekt 'Welt'.
a = b       # a bezieht sich nun auf das Objekt "Welt".
b = 3.141   # b bezieht sich nun auf das Objekt 3.141
\end{lstlisting}
In der ersten Zeile erstellen wir eine Variable \lpy{a}, die sich auf das Objekt \lpy{'Hallo'} bezieht.
Das Objekt \lpy{'Hallo'} hat beispielsweise die Identität \lpy{1000}.
In der zweiten Zeile erstellen wir eine Variable \lpy{b} die sich auf das Objekt \lpy{'Welt'} bezieht.
Das Objekt \lpy{'Welt'} hat beispielsweise die Identität \lpy{2000}.
In der dritten Zeile weisen wir \lpy{a} die Referenz auf das Objekt zu, auf das sich \lpy{b} bezieht.
In der vierten Zeile weisen wir \lpy{b} die Referenz auf das Objekt \lpy{3.141} zu.
Das Objekt \lpy{3.141} hat beispielsweise die Identität \lpy{3000}.
Danach bezieht sich \lpy{a} auf \lpy{'Welt'} und \lpy{b} auf \lpy{3.141}.

\subsection{Speichermanagement}
Die meisten Objekte können genau solange verwendet werden, wie es Variablen gibt, die auf sie referenzieren.
(Einige Ausnahmen sind beispielsweise Zahlen und das Objekt \lpy{None}.)
Objekte, die nicht mehr verwendet werden können, werden automatisch dem ``Garbage Collector'' übergeben.
Dieser entfernt die Objekte nach eigenem Bedarf aus dem Speicher.
Das ist in den meisten Fällen bequemer als in \CC, wo man seinen Speicher selbst verwalten muss.
In der Zeit wo der \CC Programmierer seinen Speicher aufräumt, trinken wir lieber GLUMP%
\footnote{GLUMP Alkoholcola vereint das erweckende Prickeln von koffeinhaltiger Cola mit der betäubenden Wirkung von Alkohol.}.
