\section{Klassen}
\label{section:klassen}

Man benutzt Klassen, um Code übersichtlicher zu gestalten.
Immer wenn wir einen (mathematischen oder realen) Gegenstand modelieren wollen, dessen Zustand
\begin{enumerate}[label=\textbf{(K\arabic*})]
  \item \label{klassenmerkmal_zustand_beschr} durch ein Verzeichniss von Eigenschaften beschrieben wird und
  \item \label{klassenmerkmal_zustand_aenderung} der Zustand durch eine oder mehrere feste Regeln verändert wird
\end{enumerate}
benutzen wir dazu sogenannte Klassen.

Bevor wir besprechen wie man Klassen in \Python definiert und nutzt, schauen wir uns ein Beispiel für eine Klasse an.
Der Zustand einer Matrix $M$ wird durch ihre Größe und ihre Koeffizienten bestimmt.
Damit erfüllen Matrizen das erste Kriterium \ref{klassenmerkmal_zustand_beschr}.
Es macht Sinn die Einträge einer Matrix abzufragen, die Einträge zu ändern und Matrizen zu addieren oder zu multiplizieren (sofern sie die richtigen Größen haben).
Damit erfüllen Matrizen das zweite Kriterium \ref{klassenmerkmal_zustand_aenderung}.
Wie man eine Matrix modelliert hängt von \ref{klassenmerkmal_zustand_beschr} und \ref{klassenmerkmal_zustand_aenderung} ab.
Allerdings legen \ref{klassenmerkmal_zustand_beschr} und \ref{klassenmerkmal_zustand_aenderung} nicht völlig fest, wie die Modellierung aussehen muss.
Zum Beispiel können Matrizen durch zwei natürliche Zahlen (welche die Anzahl der Zeilen und Spalten festlegt) sowie die Liste der Einträge beschrieben werden,
man kann aber auch das ``Compressed Sparse Row Storage'' Format verwenden.
Für die reine Benutzung von Matrizen spielt diese Designentscheidung keine Rolle.

An dieser Stelle machen wir noch darauf aufmerksam, wie man Klassen konzeptionell in \C umsetzen würde.
Da man aus \C an \lcpp{struct} gewöhnt ist, kann man eine Klasse durch ein \lcpp{struct K} zusammen mit einer Menge von Funktionen realisieren, deren erstes Argument vom Typ \lcpp{struct K*} ist.
Genau so haben wir Vektoren und Matrizen in der Bibliothek \lcpp{JoelixBlas} realisiert (siehe \cite{joelixC}).

\subsection{Klassen definieren und nutzen}
\label{section:klassen:klassen_definieren_und_nutzen}
Eine Klasse definiert man \Python wie folgt:
\begin{lstlisting}
class klassenname:
  def __init__(self, weitere_parameter):
    self.membervariable_1 = ...
    self.membervariable_2 = ...
    ...
  
  def memberfunktion_1(self, weitere_parameter):
    ...
  
  def memberfunktion_2(self, weitere_parameter):
    ...
  
  ...
  
  def __spezielle_funktion_1__(self, weitere_parameter):
    ...
  
  ...
\end{lstlisting}
Der Klassenname kann so wie alle Variablenamen fast frei gewählt werden.
Variablen die zur Klasse gehören heißen ``Membervariablen''.
Mit dem Präfix \lpy{self.} greift man auf sie {\bfseries innerhalb der Klasse} zu (aber nicht von außen).
Funktionen die zur Klasse gehören heißen ``Memberfunktionen''.
Ihr erstes Argument muss bei ihrer {\bfseries Definition innerhalb der Klasse} \lpy{self} sein, beim Zugriff wird der erste Prameter \lpy{self} nicht angegeben.

Es gibt zwei Sorten von Memberfunktionen:
sogenannte ``spezielle Funktionen'' die mit \lpy{__} beginnen und enden und
``normale Funktionen'', die nicht mit \lpy{__} beginnen und enden.
Mehr zu spezielle Funktionen besprechen wir in Abschnitt \ref{section:klassen:spezielle_funktionen}

Die prominenteste spezielle Funktion ist der sogenannte ``Konstruktor'' oder ``Initialisierungsfunktion'' mit dem Namen \lpy{__init__}.
Sie wird beim Erstellen eines Objekts automatisch aufgerufen.
Außerdem kann man beim Erstellen eines Objekts gewisse Parameter übergeben.
Bei unserem Matrixbeispiel würde es Sinn machen, die Zeilen- und Spaltengröße zu übergeben.
In der Initialisierungsfunktion sollen alle Membervariablen definiert und (in Abhängigkeit von den übergebenen Parametern) in einen sinnvollen Ausganszustand gebracht werden.
Man darf in der Initialisierungsfunktion auch andere Memberfunktionen aufrufen.
Hat der Konstruktor der Klasse \lpy{klassenname} die Parameter \lpy{p_1, ..., p_k},
so erstellt man ein Objekt vom Typ \lpy{klassenname} zu den Werten \lpy{par_1, ..., par_k} wie folgt.
\begin{lstlisting}
# Erstellt ein Objekt vom Typ klassenname bzgl. par_1, ..., par_k.
# Die Variable var referenziert auf dieses Objekt.
var = klassenname(par_1, ..., par_k)
\end{lstlisting}
Man sagt ``das Objekt \lpy{obj} ist eine Instanz der Klasse \lpy{klassenname}'' wenn der Typ des Objekts \lpy{obj} die Klasse \lpy{klassenname} ist.

Zeigt eine Variable \lpy{var} auf ein Instanz von \lpy{klassenname}, so kann man auf die Membervariablen und Memberfunktionen der Instanz mithilfe des Präfixes \lpy{var.} zugreifen.
Zeigt beispielsweise die Variable \lpy{var} auf ein Objekt dass eine Memberfunktion mit Namen \lpy{python_ist_cool},
dann greifen wir auf diese Memberfunktion mit \lpy{var.python_ist_cool()} zu.

\subsection{Eine sehr naive Matrixklasse}
\label{section:klassen:eine_sehr_naive_matrixklasse}
Als einführendes Beispiel definieren wir eine sehr naive Matrixklasse.
Dabei vernachlässigen wir alle Plausibilitätsprüfungen
(zum Beispiel prüfen wir nicht, ob der Typ von \lpy{zeilen} wirklich \lpy{int} ist).
Bei einer gut geschriebenen Klasse dürfen diese Plausibilitätsprüfungen natürlich nicht fehlen.
\begin{lstlisting}
class Matrix:
  """Eine sehr naive Matrixklasse"""
  
  def __init__(self, zeilen=0, spalten=0):
    self.zeilen = zeilen
    self.spalten = spalten
    self.elemente = self.zeilen*self.spalten*[0.0]
  
  def getitem(self,i,j):
    """Gibt den Koeffizienten der Matrix in der i-ten Zeile und
       der j-ten Spalte zurueck."""
    return self.elemente[i*self.zeilen + j]
  
  def setitem(self, i,j ,z):
    """Setzt den Koeffizienten der Matrix in der i-ten Zeile und
       der j-ten Spalte auf den Wert z."""
    self.elemente[i*self.zeilen + j] = z
  
  def string(self):
    """Gibt einen String zurueck, der die Matrix beschreibt."""
    s = "Zeilen: {} Spalten: {}\n".format(self.zeilen, self.spalten)
    for i in range(self.zeilen):
      for j in range(self.spalten):
        s = s + "{}; ".format(self.getitem(i,j))
      s = s + "\n"
    return s
\end{lstlisting}
Mit dieser sehr naiven Definition, erstellt man eine Matrix mit drei Zeilen und vier Spalten wie folgt:
\begin{lstlisting}
m = Matrix(zeilen=3, spalten=4)
\end{lstlisting}
Auf die Memberfunktionen greift man mit dem Präfix \lpy{m.} zu:
\begin{lstlisting}
m.setitem(1,1,3.0) # Setzt den Eintrag in Zeile 1 und Spalte 1 auf 3.0.
print(m.string())  # Druckt die Darstellung der Matrix aus.
\end{lstlisting}

\subsection{Spezielle Funktionen}
\label{section:klassen:spezielle_funktionen}
In diesem Abschnitt erklären wir zunächst was ``Duck-Typing'' ist,
listen die wichtigsten speziellen Memberfunktionen auf und
verbessern unser Matrizenbeispiel aus Abschnitt \ref{section:klassen:klassen_definieren_und_nutzen}.


\subsubsection{Duck-Typing}
Als ``Duck-Typing'' bezeichnet man das Konzept, dass ein Objekt durch sein Verhalten und nicht durch seinen Bauplan beschrieben wird.
Der Name Duck-Typing ist an ein Gedicht von James Rileys angelehnt:
\begin{center}
\begin{minipage}{.8\columnwidth}
\centering
When I see a bird that walks like a duck and swims like a duck and quacks like a duck, I call that bird a duck.
\end{minipage}
\end{center}
Auch \Python unterstützt Duck-Typing.
Zum Beispiel können viele Klassen als \lpy{bool} oder \lpy{str} interpretiert werden.
Das funktioniert genau dann, wenn die Klasse die spezielle Memberfunktionen \lpy{__str__} implementiert und
diese einen String zurückgibt.
Dann kann man mit der Funktion \lpy{str} das Objekt als String interpretieren.

Wenn wir uns unsere sehr naive Matrixklasse aus Abschnitt \ref{section:klassen:eine_sehr_naive_matrixklasse} nochmal anschauen,
so kann \Python diese momentan noch nicht als String interpretieren (denn sie implementiert spezielle Memberfunktionen \lpy{__str__} noch nicht).
Allerdings kann sie als String interpretiert werden, wenn wir den Namen der Memberfunktion \lpy{string} durch \lpy{__str__} ersetzen.
Soland wir das erledigt haben, kann \Python den folgenden Code interpretieren:
\begin{lstlisting}
m = Matrix(zeilen=3, spalten=4)
print(m)  # Druckt die Darstellung der Matrix aus.
\end{lstlisting}

Die Funktion \lpy{str} wird vom \Python-Standard definiert und könnte ungefähr so aussehen:
\begin{lstlisting}
def my_str( t ):
  """Nachbau der eingebauten Funktion 'str'."""
  s = t.__str__()    # Hole den String
  if type(s) is str: # Schaue ob s den richtigen Typ hat
    return s         # Gibt den String s zurueck
  else:              # Fehlerbehandlung
    typename = type(s).__name__
    raise TypeError("__str__ returned non-string (type "+typename+")")
\end{lstlisting}


\subsubsection{Spezielle Memberfunktionen}
In diesem Abschnitt besprechen wir kurz die wichtigsten speziellen Memberfunktionen.
Dabei machen wir optionale Funktionsparameter mit \lpy{[...]} kenntlich.

Der Konstruktor wird beim Erstellen eines Objekts aufgerufen und soll das Objekt in Ausgangszustand bringen.
Dem Konstruktor kann man weitere Parameter übergeben.
\begin{lstlisting}
__init__(self [,...])  # Konstruktor (wird beim Erstellen ausgefuehrt)
\end{lstlisting}
Ist der Konstruktor definiert, kann man ein Objekt der Klasse \lpy{K} in Abhängigkeit von den Parametern \lpy{p_1, ..., p_r} so erstellen:
\begin{lstlisting}
var = K(p_1, ..., p_r)
\end{lstlisting}

Auf den Destuktor gehen wir hier nicht ein, da wir das sogenannte ``Reference Counting'' und den sogenannten ``Garbadge Collector''
in diesem Skript nicht ausreichend behandelt haben.

Ein Objekt kann als String, Ganzzahl, Gleitkommazahl bzw.\ Boolean interpretiert werden,
sofern die entsprechende spezielle Funktion definiert ist.
\begin{lstlisting}
__str__(self)   # Konvertiert Objekt zu String
__int__(self)   # Konvertiert Objekt zu Integer
__float__(self) # Konvertiert Objekt zu Float
__bool__(self)  # Konvertiert Objekt zu Boolean
\end{lstlisting}
Ist beispielsweise \lpy{__bool__} implementiert, kann man ein Objekt \lpy{obj} mit \lpy{bool(obj)} als Boolean interpretieren.

Die nachfolgenden speziellen Funktionen ermöglichen es, zwei Objekte miteinander zu vergleichen.
\begin{lstlisting}
__lt__(self, other) # Implementiert: obj <  other
__le__(self, other) # Implementiert: obj <= other
__eq__(self, other) # Implementiert: obj == other
__ne__(self, other) # Implementiert: obj != other
__gt__(self, other) # Implementiert: obj >  other
__ge__(self, other) # Implementiert: obj >= other
\end{lstlisting}
Hier ist es ganz wichtig darauf zu achten, dass diese Funktion konsistent implementiert werden.
Zum Beispiel will man, dass wenn \lpy{obj == other} gilt dann auch \lpy{obj <= other} erfüllt ist.
Außerdem muss man ganz doll aufpassen, wenn man zwei Objekte vergleichen möchte, die einen verschiedenen Typ haben.
Wenn man das nicht möchte, kann man gegebenfalls eine Ausnahme auslösen:
\begin{lstlisting}
class K:
  ... # Konstruktor und weitere Funktionen
  def __eq__(self, other):         # Definition der Fkt __eq__
    if type(self) == type(other):  # Wenn die Typen gleich sind
      ...                          # Pruefe Objekte auf Gleichheit
    else:                          # Sonst:
      raise TypeError("Typen sind verschieden") # Loese Ausnahme aus
  ... # Weitere Funktionen
\end{lstlisting}
An dieser Stelle wollen wir darauf aufmerksam machen, dass das Vergleichen der Typen von zwei Objekten, so wie im obigen Beispiel, bei abgeleiteten Klassen zu ungewollten Effekten führen kann.
Da wir abgeleitete Klassen in diesem Skript nicht besprechen, gehen wir nicht näher auf diese Bemerkung ein.

Um ein Objekt als Schlüssel für ein Verzeichniss machen zu können, muss es ``hashbar'' sein.
Genauer muss man die spezielle Funktion \lpy{__hash__} implementieren.
Das besprechen wir ausführlicher im Abschnitt \ref{section:klassen:hashbare_objekte}.
\begin{lstlisting}
__hash__(self)  # Objekt hashen
\end{lstlisting}

Um ein Objekt \lpy{obj} wie eine Funktion aufzurufen, also um dem Ausdruck \lpy{obj(...)} interpretierbar zu machen,
muss die spezielle Funktion \lpy{__call__} implementiert werden.
Ein Objekt, welches sich in diesem Sinne wie eine Funktion verhält, nennt man \emph{Funktionsobjekt}.
\begin{lstlisting}
__call__(self,[, args...]) # Objekt wird zum Funktionsobjekt
\end{lstlisting}

Um ein Objekt \lpy{obj} wie ein Verzeichniss (siehe Abschnitt \ref{section:std_data_types:verzeichnisse}) zu behandeln, implementiert man die folgenden speziellen Funktionen:
\begin{lstlisting}
__len__(self)                 # Laenge des Objekts
__getitem__(self, key)        # Implementiert: obj[key]
__setitem__(self, key, value) # Implementiert: obj[key] = value
__delitem__(self, key)        # Implementiert: del obj[key]
__reversed__(self)            # Implementiert: reversed(obj)
__contains__(self, item)      # Implementiert: is item in obj
\end{lstlisting}
An dieser Stelle machen wir darauf aufmerksam, welche Werte der Schlüssel \lpy{key} haben kann.
Ein Schlüssel muss immer ein hashbares Objekt sein.
Im einfachsten Beispiel sind das unveränderbare Standardtypen wie \lpy{int} oder \lpy{str}.
Es können aber auch benutzerdefinierte Klassen sein, welche die spezielle Funktion \lpy{__hash__} implementieren (siehe auch Abschnitt \ref{section:klassen:hashbare_objekte}).
In diesen beiden Beispielen wird \lpy{obj[key] = z} so interpretiert:
\begin{lstlisting}
obj.__setitem__(key,z)    # Aequivalent zu obj[key] = z
\end{lstlisting}
Man kann aber auch mehrere Schlüssel zum Zugriff benutzen.
Diese werden dann zu einem einzigen Tupel gebündelt.
Schauen uns zum Beispiel an, wie \lpy{obj[key1,key2] = z} interpretiert wird:
\begin{lstlisting}
obj.__setitem__( (key_1,key_2) ,z) # Aequivalent zu obj[key1,key2] = z
\end{lstlisting}
In unserem Matrixbeispiel aus Abschnitt \ref{section:klassen:eine_sehr_naive_matrixklasse},
würden wir sehr gern mit \lpy{M[i,j]} den Koeffizient in der $i$-ten Zeile und der $j$-ten Spalte setzen.
Die spezielle Memberfunktionen \lpy{__setitem__} kann so implementiert werden:
\begin{lstlisting}
class Matrix:
  """Eine native Matrixklasse"""
  ...           # Konstruktor und weitere Funktionen
  def __setitem__(self, ij ,z):
    i,j = ij    # Wir interpretieren ij als Tupel ij = (i,j)
    self.elemente[i*self.zeilen + j] = z # Setze Koeffizient
  ...           # Weitere Funktionen
\end{lstlisting}

In \Python ist es (genauso wie in \CPP) möglich aus zwei Objekten mithilfe der binären Operatoren
\lpy{+}, \lpy{-}, \lpy{*} usw.\ ein neues Objekt zu erstellen.
Dazu implementiert man die folgenden speziellen Memberfunktionen.
Wir machen hier darauf aufmerksam, dass die speziellen Memberfunktionen die beiden Objekte \lpy{obj} und \lpy{other}
in so gut wie allen Fällen nicht veränderen, da man aus den beiden Objekten \lpy{obj} und \lpy{other}
mithilfe des Operators \lpy{op} ein neues Objekt \lpy{obj op other} erstellt.
\begin{lstlisting}
__add__(self, other)      # Implementiert: obj  + other
__sub__(self, other)      # Implementiert: obj  - other
__mul__(self, other)      # Implementiert: obj  * other
__truediv__(self, other)  # Implementiert: obj  / other
__floordiv__(self, other) # Implementiert: obj // other
__mod__(self, other)      # Implementiert: obj  % other
__pow__(self, other)      # Implementiert: obj ** other
__lshift__(self, other)   # Implementiert: obj << other
__rshift__(self, other)   # Implementiert: obj >> other
__and__(self, other)      # Implementiert: obj  & other
__xor__(self, other)      # Implementiert: obj  ^ other
__or__(self, other)       # Implementiert: obj  | other
\end{lstlisting}

Es ist wichtig zu bemerken, dass keine die obigen binären Operationen symmetrisch ist:
\begin{lstlisting}
obj1.__add__(obj2)        # obj1 + obj2
obj2.__add__(obj1)        # obj2 + obj1
\end{lstlisting}
Es kommt vor, dass man zwei Objekte von verschiedenen Typen miteinander verbinden möchte.
Hier gibt es zwei Möglichkeiten.
Entweder man implementiert die binären Operationen für beide Klassen oder man implementiert sie nur für eine.
Letzteres macht in vielen Fällen mehr Sinn, zum Beispiel wenn eine der beiden Klassen schon von jemand anderem implementiert ist.
Wie Letzteres funktioniert erklären wir jetzt.

Nehmen wir an, wir wollen zwei Objekte \lpy{obj1} und \lpy{obj2} mit einer binären Operation \lpy{op} verbinden,
i.e.\ wir wollen den Ausdruck \lpy{obj1 op obj2} interpretierbar machen.
Wir verlangen dass
(a) die beiden Objekte verschiedene Typen haben und
(b) das linke Objekt \lpy{obj1} die spezielle Memberfunktion \lpy{__op__} nicht implementiert hat.
Dann können wir \lpy{obj1 op obj2} durch die spezielle Memberfunktion \lpy{__rop__} vom rechten Objekt \lpy{obj2} interpretierbar machen.
\begin{lstlisting}
obj2.__rop__(obj1)         # obj1 op obj2 falls (a) und (b) gelten
\end{lstlisting}

Die folgende Liste erklärt die binären Operationen, die wir im Sinne des vergangenen Abschnitts interpretierbar machen können.
Auch hier machen wir darauf aufmerksam, dass diese binären Operationen nach Möglichkeit ein neues Objekt erstellen sollen ohne dabei die Operanden zu verändern.
\begin{lstlisting}
__radd__(self, other)      # Implementiert: other  + obj
__rsub__(self, other)      # Implementiert: other  - obj
__rmul__(self, other)      # Implementiert: other  * obj
__rtruediv__(self, other)  # Implementiert: other  / obj
__rfloordiv__(self, other) # Implementiert: other // obj
__rmod__(self, other)      # Implementiert: other  % obj
__rpow__(self, other)      # Implementiert: other ** obj
__rlshift__(self, other)   # Implementiert: other << obj
__rrshift__(self, other)   # Implementiert: other >> obj
__rand__(self, other)      # Implementiert: other  & obj
__rxor__(self, other)      # Implementiert: other  ^ obj
__ror__(self, other)       # Implementiert: other  | obj
\end{lstlisting}

Wir wissen ja bereits, dass \lpy{c = a + b} ein neues Objekt aus \lpy{a} und \lpy{b} erstellt und dass sich die Variable \lpy{c} im Anschluss darauf bezieht.
Insbesondere bleiben die Objekte \lpy{a} und \lpy{b} unverändert.
Dasslbe gilt für \lpy{a = a + b}.
Insbesondere ist \lpy{a = a + b} etwas anderes als \lpy{a += b}.
Das haben wir zum Beispiel bei Listen im Abschnitt \ref{section:std_data_types:sequenzen:listen} gesehen.
Mit den folgenden speziellen Memberfunktionen realisiert man die Operationen der Form \lpy{a += b}, \lpy{a -= b}, usw..
\begin{lstlisting}
__iadd__(self, other)      # Implementiert: obj  += other
__isub__(self, other)      # Implementiert: obj  -= other
__imul__(self, other)      # Implementiert: obj  *= other
__itruediv__(self, other)  # Implementiert: obj  /= other
__ifloordiv__(self, other) # Implementiert: obj //= other
__imod__(self, other)      # Implementiert: obj  %= other
__ipow__(self, other)      # Implementiert: obj **= other
__ilshift__(self, other)   # Implementiert: obj <<= other
__irshift__(self, other)   # Implementiert: obj >>= other
__iand__(self, other)      # Implementiert: obj  &= other
__ixor__(self, other)      # Implementiert: obj  ^= other
__ior__(self, other)       # Implementiert: obj  |= other
\end{lstlisting}

Wie man den Absolutbetrag und die uniären Operationen \lpy{-ob}, \lpy{+ob} und \lstinline[style=Pyinline]|~ob| interpretierbar macht,
erklärt die nachfolgende Liste.
\begin{lstlisting}
__abs__(self, other)      # Absolutbetrag des Objekts
__neg__(self, other)      # Implementiert: -obj
__pos__(self, other)      # Implementiert: +obj
__invert__(self, other)   # Implementiert: ~obj
\end{lstlisting}

\subsection{Eine naive Matrixklasse}
\label{section:klassen:eine_naive_matrixklasse}
Wir verbessern hier unsere sehr naive Matrixklasse aus Abschnitt \ref{section:klassen:eine_sehr_naive_matrixklasse}
indem wir die sehr naiven Memberfunktionen durch spezielle Memberfunktionen ersetzen.
Dadurch ist der Zugriff auf die Koeffizienten einer Matrix und die Ausgabe einer Matrix viel komfortabler.
\begin{lstlisting}
class Matrix:
  """Eine naive Matrixklasse"""
  
  def __init__(self, zeilen=0, spalten=0):
    self.zeilen = zeilen
    self.spalten = spalten
    self.elemente = self.zeilen*self.spalten*[0.0]
  
  def __getitem__(self,ij):
    """Implementiert den Zugriffsoperator []. Mit m[i,j] wird der
       Koeffizienten der Matrix m in der i-ten Zeile und der j-ten Spalte
       zurueckgegeben."""
    i,j = ij
    return self.elemente[i*self.zeilen + j]
  
  def __setitem__(self, ij ,z):
    """Implementiert den Zugriffsoperator []. Mit m[i,j] = z wird der
       Koeffizienten der Matrix m in der i-ten Zeile und der j-ten Spalte
       auf den Wert z gesetzt."""
    i,j = ij
    self.elemente[i*self.zeilen + j] = z
  
  def __str__(self):
   """Implementiert die Konvertierung zu str. Mit str(m) wird ein String
      zurueckgegeben, der die Matrix m beschreibt."""
    s = "Zeilen: {} Spalten: {}\n".format(self.zeilen, self.spalten)
    for i in range(self.zeilen):
      for j in range(self.spalten):
        s = s + "{}; ".format(self[i,j])
      s = s + "\n"
    return s
\end{lstlisting}
Mit dieser verbesserten Implementierung der Klasse \lpy{Matrix} können wir viel komfortabler mit Matrizen arbeiten.
\begin{lstlisting}
m = Matrix(zeilen=3,spalten=4) # Erstellt eine 3x4 Matrix
m[1,1] = 3.0  # Setzt den Eintrag in Zeile 1 und Spalte 1 auf 3.0.
print(m)      # Druckt die Darstellung der Matrix aus.
\end{lstlisting}

\subsection{Hashbare Objekte}
\label{section:klassen:hashbare_objekte}
Wir erklären hier, was hashbare Objekte sind und wie man selbstdefinierte Klassen hashbar macht.

Hashfunktionen und Hashtables wurden im Fortgeschrittenen Programmierkurs (siehe \cite{joelixC}) bereits erklärt, deshalb halten wir uns hier kurz.
Für einen festgelegten Typ \lpy{T} nennen wir Menge aller Objekte diesen Typs $Set_T$.
Die Menge der \Python-Ganzzahlen nennen wir hier $Int$.
Man sagt dass
\[
  hash \colon Set_T \to Int
\]
eine ``zulässige Hashfunktion'' ist, wenn für je zwei Objekte \lpy{x} und \lpy{y} aus $Set_T$ für die \lpy{x == y} als \lpy{True} auswertet auch \lpy{hash(x) == hash(y)} als \lpy{True} auswertet.
Desweiteren muss bei mutable Objekten gewähleistet sein, dass beim Ändern des Werts eines jeden Objekts \lpy{x} die Ausdücke \lpy{x == y} und \lpy{hash(x) == hash(y)} unabhängig von der Änderung sind.
Man sagt, dass ein Objekt oder ein Typ \lpy{T} ``hashbar'' ist, wenn eine zulässige Hashfunktion für $Set_T$ definiert ist.
Der \Python-Standard hält für die dort definierten immutable Typen (also Objekte vom Typ \lpy{int}, \lpy{float}, \lpy{complex}, \lpy{tuple}, \lpy{str}, \lpy{bool} und \lpy{none},
aber nicht \lpy{list} und \lpy{dict}) zulässige Hashfunktionen bereit.
Demnach sind diese Typen hashbar.

Um eine selbstgeschriebene Klasse \lpy{K} hashbar zu machen, müssen wir die beiden spezielle Funktion \lpy{__eq__} und \lpy{__hash__} implementieren.
Dabei muss (wir oben gefordert) für je zwei Instanzen \lpy{x} und \lpy{y} von \lpy{K} für die \lpy{x == y} als \lpy{True} auswertet, der Wert von \lpy{x.__hash__()} und \lpy{y.__hash__()} übereinstimmen.
Außerdem verlangen wir, dass, falls das Objekt (durch Memberfunktionen) verändert werden kann, so soll \lpy{__eq__} und \lpy{__hash__} unabhängig von dieser Änderung sein.

Hier ein ganz simples Beispiel.
Wir beschreiben eine Klasse mit dem Namen \lpy{Geldboerse}, inder man 1-Euro Münzen und 1-Cent Münzen speichern kann.
Die Anzahl der Münzen soll nach dem Erstellen nicht mehr geändert werden.
Zwei Geldbörsen sind genau dann gleich, wenn sie dieselbe Anzahl von 1-Euro und 1-Cent Münzen enthalten.
Als Hashfunktion nehmen wir die Anzahl der 1-Euro Münzen modulo 100.
Wir bemerken, dass das eine zulässige Hashfunktion ist.
Der Code sieht dann so aus:
\begin{lstlisting}
class Geldboerse:
  """Eine sehr naive Klasse die eine Geldboerse darstellt."""
  def __init__(self, euro, cent):
    self._euro = euro
    self._cent = cent
  
  def __str__(self): # Macht Geldboerse als String interpretierbar
    return 'Euro: {}, Cent: {}'.format(self._euro, self._cent)
  
  def __eq__(self, other): # Implementiert x == y
    if self._euro == other._euro and self._cent == other._cent:
      return True
    else:
      return False
  
  def __hash__(self): # Eine sehr naive Hashfunktion
    return self._euro % 100
\end{lstlisting}

Wenn wir nun doch zulassen, dass man die Anzahl der Münzen geändert werden kann,
so wäre unsere oben angegebene Hashfunktion nicht mehr zulässig.
Es ist also gar nicht so leicht, eine zulässige Hashfunktion für beliebige Klassen zu schreiben und in der Tat,
man will meistens nur solche Objekte hashbar machen, die sich nach dem Erstellen nur unwesentlich oder garnicht mehr verändern.


