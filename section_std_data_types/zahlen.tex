\subsection{Zahlen}
\label{section:std_data_types:zahlen}
In \Python gibt es die numerischen Typen \lpy{int}, \lpy{bool}, \lpy{float} und \lpy{complex}.
Objekte dieses Typs sind immer ``immutable''.

Die \Python-Typen \lpy{int}, \lpy{float} und \lpy{complex} verhalten sich ähnlich wie die \CC-Typen \lcpp{int}, \lcpp{float} und \lcpp{complex}.
Hier die wichtigsten Operationen.
\begin{lstlisting}
x + y  # Summe von x und y
x - y  # Differenz von x und y
x * y  # Produkt von x und y
x / y  # Quotient von x and y
x // y # Abgerundeter Quotient x und y
x % y  # Rest von x / y
-x     # Negiere x
+x     # Tut nichts
x ** y # x hoch y
abs(x) # Betrag von x
int(x) # Erstellt int aus x
float(x)        # Erstellt float aus x
complex(re, im) # Erstellt complex aus re und im
divmod(x, y)    # Das Paar (x // y, x % y)
\end{lstlisting}
Die Division von zwei Ganzzahlen wieder eine Ganzzahl in \PythonZwei und eine Gleitkommazahl in \PythonDrei.
Sind die Typen der Operationen verschieden, verhält sich \Python ebenfalls wie \CC und nimmt immer den genaueren Typ.
\begin{lstlisting}
2   /  3   # = 0 in Python2; 0.6666666666666666 in Python3
2   /  3.0 # = 0.6666666666666666
2.0 /  3   # = 0.6666666666666666
2.0 // 3.0 # = 0.0
\end{lstlisting}

Des Weiteren gibt es für \lpy{int}, \lpy{float} und \lpy{complex} typenspezifische Funktionen.
Für \lpy{int} gibt es beispielsweise Bitoperationen \lstinline[style=PyInline]|~|, \lpy{|}, \lpy{&}, \lstinline[style=PyInline]|^|, \lpy{<<} und \lpy{>>};
für \lpy{float} gibt es beispielsweise die Funktion \lpy{as_integer_ration}, die \lpy{x} als Bruch darstellt;
und komplexe Zahlen können beispielsweise mit der Funktion \lpy{conjugate} ihr Komplexkonjugiertes zurückgeben.
Mehr solcher Funktionen sind hier zu finden \cite[Library, Build-in Types, Numeric Types]{Python3}.

Die Funktion \lpy{float(x)} erstellt aus \lpy{x} ein Objekt vom Typ \lpy{float}.
Praktischerweise ist der Typ des Objekts \lpy{x} fast unbeschränkt.
Hier ein Beispiel.
\begin{lstlisting}
float(700)           # Erzeugt das Objekt 700.0
float('   -12345\n') # Erzeugt das Objekt -12345.0
float('1e-003')      # Erzeugt das Objekt 0.001
float('+1E6')        # Erzeugt das Objekt 1000000.0
float('-Infinity')   # Erzeugt das Objekt -inf
\end{lstlisting}
An dieser Stelle sei noch gesagt, dass ein Objekt vom Typ \lpy{typ} genau dann als \lpy{float} (bzw.\ \lpy{int} oder \lpy{complex}) interpretiert werden,
wenn die zu \lpy{typ} gehörige Klasse über die speziellen Memberfunktion \lpy{__float__} (bzw.\ \lpy{__int__} oder \lpy{__complex__}) verfügt.
Was das genau heißt, wird im Abschnitt \ref{section:klassen} erklärt.
