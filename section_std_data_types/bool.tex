\subsection{bool}
\label{section:std_data_types:bool}
Es gibt genau zwei Objekte vom Typ \lpy{bool}.
Sie sind durch ihren Wert \lpy{True} und \lpy{False} eindeutig bestimmt.
Außerdem sind sie immutable.

Als \lpy{False} interpretiert \Python die folgende Werte:
Zahlen \lpy{0} und \lpy{0.0};
leere Mengen (ja, leere Menge{\bfseries n}, i.e.\ Objekte vom Typ \lpy{set}, deren Wert mit dem Wert \lstinline[style=Pyinline]|{}| übereinstimmt (selbst wenn ihre Identitäten verschieden sind));
leere Sequenzen \lpy{''}, \lpy{()} und \lpy{[]};
und leere Verzeichnisse.
Allgemein kann man ein Objekt vom Typ \lpy{x} genau dann als \lpy{bool} interpretiert werden, wenn die zu \lpy{x} gehörige Klasse über eine Memberfunktion \lpy{__bool__} verfügt.
Was das genau heißt, wird im Abschnitt \ref{section:klassen} erklärt.

Es gibt folgende boolschen Operationen.
\begin{lstlisting}
x or  y # y wird nur ausgewertet wenn x False ist
x and y # y wird nur ausgewertet wenn x True ist
  not x # False wenn x True ist und True wenn x False ist
\end{lstlisting}
