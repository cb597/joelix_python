\subsection{Sequenzen}
\label{section:std_data_types:sequenzen}
Sequenzen sind sogenannte ``Container'', die sich zum Teil ähnlich wie die aus \CC bekannten Arrays verhalten.
Die wichtigsten Sequenztypen sind der mutable Typ \lpy{list}, sowie die immutable Typen \lpy{tuple} und \lpy{str}.


\subsubsection{Gemeinsamkeiten}
\label{section:std_data_types:sequenzen:gemeinsamkeiten}
Besprechen wir zunächst die wichtigsten Operationen, die sich die drei Typen \lpy{list}, \lpy{tuple} und \lpy{str} teilen.
\begin{lstlisting}
a = [x,y,z] # a ist Typ list  mit a[0]=x, a[1]=y, a[2]=z
a = (x,y,z) # a ist Typ tuple mit a[0]=x, a[1]=y, a[2]=z
a = 'xyz'   # a ist Typ str   mit a[0]=x, a[1]=y, a[2]=z
a[i]        # Die Variable a[i]
len(a)      # Laenge von a
\end{lstlisting}
Für nicht-negative Ganzzahlen \lpy{i} ist völlig klar was damit gemeint ist.
Allerdings sind auch negative Werte für \lpy{i} zulässig: hier benutzt \Python den Index \lpy{len(a) + i}.
Insgesamt sind nur Werte \lpy{-len(a) <= i < len(a)} zulässig.

Um zu überprüfen, ob (der Wert des Objekts der Variable) \lpy{x} mindestens einmal in \lpy{a} vorkommt oder nicht, gibt es die folgenden Operationen.
\begin{lstlisting}
x in a      # True  genau dann wenn x in a enthalten ist
x not in a  # False genau dann wenn x in a enthalten ist
\end{lstlisting}

Um zwei Container desselben Typs (also \lpy{list}, \lpy{tuple} oder \lpy{str}) hintereinander zu heften gibt es folgende Operationen.
Dabei entsteht immer ein neues Objekt.
\begin{lstlisting}
a + b # Erstelle die Verkettung von a und b (erst a dann b)
n * a # a + a + ... + a (n mal)
\end{lstlisting}
Beispielsweise erstellt man eine Liste der Größe \lpy{n} mit Einträgen \lpy{0} wie folgt.
\begin{lstlisting}
thor = n*[0] # Die Sequenz [0, 0, ...., 0] der Laenge n
\end{lstlisting}

Einen Teilsequenz kann man wie folgt bekommen.
\begin{lstlisting}
b=a[i:j]   # b[t] = a[i+t]   mit i <= i+t   < j
b=a[i:j:k] # b[t] = a[i+t*k] mit i <= i+t*k < j
\end{lstlisting}
Für nicht-negative \lpy{i} und \lpy{j} ist völlig klar was damit gemeint ist.
Auch hier sind negative Werte für \lpy{i} und \lpy{j} zulässig solange \lpy{-len(a) <= i,j < len(a)} erfüllt ist.
\begin{lstlisting}
a[2:-4] # Die Variablen a[2], ..., a[n-4] fuer n = len(a)
\end{lstlisting}
Außerdem können \lpy{i} und \lpy{j} weggelassen werden.
Dann ist \lpy{i = 0} bzw.\ \lpy{j = len(a)}.
\begin{lstlisting}
a[ :-4] # Die Variablen a[0], ..., a[n-4] fuer n = len(a)
a[2:  ] # Die Variablen a[2], ..., a[n]   fuer n = len(a)
\end{lstlisting}
Bei diesen Teilsequenzen ist es wichtig zu wissen, dass hier sogenannte ``oberflächeliche Kopien'' erstellt werden.
Was das ist und welche (ungeahnten) Konsequenzen das haben kann, behandeln wir im Abschnitt \ref{section:std_data_types:sequenzen:kopien}.

Weitere nützliche Operationen sind die Folgenden.
\begin{lstlisting}
min(a)     # Kleinstes Element von a, falls sinnvolle Frage
max(a)     # Groesstes Element von a, falls sinnvolle Frage
a.count(x) # Anzahl der Variablen die gleich x sind.
a.index(x) # Index von der ersten Variablen, die gleich x ist
a.index(x, i)    # Index von der ersten Variablen, die gleich x ist sowie
                 # gleich oder nach i kommt
a.index(x, i, j) # Index von der ersten Variablen, die gleich x ist sowie
                 # gleich oder nach i sowie vor j kommt
\end{lstlisting}


\subsubsection{Oberflächeliche und tiefe Kopien}
\label{section:std_data_types:sequenzen:kopien}
Eine Sequenz ist ein Objekt.
Der Wert einer Sequenz ist eine linear geordnete Menge von Variablen.
Ja richtig, der Wert einer Sequenz ist eine linear geordnete Menge von Variablen (und nicht von Objekten).
Deshalb ist eine sogenannte ``oberflächliche Kopie'' einer Sequenz \lpy{a} ein neues Objekt \lpy{b} mit dem selben Wert wie \lpy{a}.
Das heißt, \lpy{a[0]} und \lpy{b[0]} zeigen auf dasselbe Objekt.
Schauen wir uns das folgende Beispiel an:
\begin{lstlisting}
# In diesem Beispiel bezeichnen wir Listen der Form [i,j] wobei i und j
# Ganzzahlen sind als Vektoren.

# Zuerst erstellen wir eine Liste von drei Vektoren und kopieren die Liste
vekt_lst = [ [0,1], [2,3], [5,1] ] # Liste von drei Vektoren
kopie    = vekt_lst[0:3]           # Oberflaechliche Kopie

# Die Listen sind verschieden
print(id(vekt_lst)) # 140541651867528
print(id(kopie))    # 140541679766536

# Die Elemente von einer der Listen zu aendern hat folgenden Effekt
v = kopie[0] # zeigt auf denselben Vektor wie kopie[0] und vekt_lst[0]
v[0] = 100   # v wird geaendert
print(v)        # [100,1]
print(kopie)    # [[100, 1], [2, 3], [5, 1]]
print(vekt_lst) # [[100, 1], [2, 3], [5, 1]]
\end{lstlisting}

Dieses Verhalten ist in den meisten Fällen gewollt.
Wenn man eine ``tiefe Kopie'' der Sequenz \lpy{a} braucht, also eine Sequenz \lpy{b},
deren ``mutable'' Objekte denselben Wert, aber eine verschiedene Identität haben, dann benutzt man das Modul \lpy{copy}
(siehe auch Abschnitt \ref{section:module:empfohlene_module:copy}).
Das geschieht wie folgt.
\begin{lstlisting}
# In diesem Beispiel bezeichnen wir Listen der Form [i,j] wobei i und j
# Ganzzahlen sind als Vektoren.

import copy

# Zuerst erstellen wir eine Liste von drei Vektoren und kopieren die Liste
vekt_lst = [ [0,1], [2,3], [5,1] ] # Liste von drei Vektoren
kopie    = copy.deepcopy(vekt_lst) # Tiefe Kopie

# Die Listen sind verschieden
print(id(vekt_lst)) # 140541651867528
print(id(kopie))    # 140541679766536

# Die Elemente von einer der Listen zu aendern hat folgenden Effekt
v = kopie[0] # zeigt auf denselben Vektor wie kopie[0]
v[0] = 100   # v wird geaendert
print(v)        # [100,1]
print(kopie)    # [[100, 1], [2, 3], [5, 1]]
print(vekt_lst) # [[  0, 1], [2, 3], [5, 1]]
\end{lstlisting}


\subsubsection{tuple}
\label{section:std_data_types:sequenzen:tuple}
Der Containertyp \lpy{tuple} ist immutable. Das bedeutet, dass weder einzelne Einträge des Tupels verändert werden können, 
noch kann die Länge des Tupels verändert werden. Das Tupel \lpy{(3, 'Harry Potter')} beispielsweise wird über den gesamten Verlauf des 
Programs das Tupel \lpy{(3, 'Harry Potter')} bleiben und kann nicht zu z.B.\ \lpy{(1, 'LotR')} geändert werden. 

Dennoch nutzt man Tupel implizit rechts häufig. Unter anderem können, wie im Crashkurs Abschnitt \ref{section:crashkurs:funktionen} bereits angedeutet, Funktionen 
mehrere Objekte zurückgeben. Diese werden dann zusammen in ein Tupel gepackt, in der Reihenfolge, in der sie nach dem 
\lpy{return}-Statement aufgezählt sind. Das folgende Beispiel veranschaulicht dies:

\begin{lstlisting}
def doppeltupelfkt(x,y)
  return 2*x, 2*y

z = doppeltupelfkt(1,2)
print( z )              # gibt (2,4) aus
\end{lstlisting}

\lpy{z} ist nach Aufrufen der Funktion das Tupel \lpy{(2,4)}, d.h.\ wenn eine Funktion mehrere Objekte zurückgibt, werden sie in 
einem Objekt abgefangen.

Außerdem gibt es für Tupel noch die Möglichkeit der mehrfachen Zuordnung:

\begin{lstlisting}
x_1, x_2, ... , x_n = z # z ist Tupel mit len(z) = n
\end{lstlisting}

Hierbei wird \lpy{z[i-1]} der Variable \lpy{x_i} zugeordnet. Auf diese Weise können auch die Objekte, die eine Funktion 
zurückgibt, in mehreren Variablen abgespeichert werden statt in nur einem Tupel. Dabei muss darauf geachtet werden, dass 
auf der linken Seite der Zuordnung genau so viele Variablen stehen wie das Tupel Elemente hat.

Ansonsten verfügt \lpy{tuple} über keine weiteren (für diesen Kurs bedeutsamen) Operationen --- abgesehen von den Gemeinsamkeiten mit 
anderen Sequenzen, die im vorangegangenen Abschnitt \ref{section:std_data_types:sequenzen:gemeinsamkeiten} beschrieben wurden.


\subsubsection{list}
\label{section:std_data_types:sequenzen:listen}
Den mutable Containertyp \lpy{list} verwendet man in \Python recht oft.
Neben den Operationen, die alle Sequenzen gemeinsam haben, können wir eine gegebene Liste \lpy{a} mit den nachfolgenden Operationen verändern.

Folgende Operationen fügen Objekte hinzu oder entfernen sie.
\begin{lstlisting}
a[i] = x       # Ersetze a[i] durch x
a.insert(i, x) # Fuege x hinter a[i] ein
a.append(x)    # Fuege x am Ende von a ein
x = a.pop([i]) # Erhalte x = a[i] und entferne a[i] aus der list
a.remove(x)    # Entferne die erste Variable mit Wert x
\end{lstlisting}

Folgende Operationen verlängern eine gegebene \lpy{list} mit Namen \lpy{a} um eine (andere) gegebene \lpy{list}.
\begin{lstlisting}
a += t # Verlaengert a mit t
a *= n # Verlaengert a n mal mit sich selbst
\end{lstlisting}
Wir machen darauf aufmerksam, dass \lpy{a+=t} nicht dasselbe ist wie \lpy{a=a+t}.
Der erste Ausdruck verlängert \lpy{a} um \lpy{t}, der zweite Ausdruck erstellt ein neues Objekt mit Wert \lpy{a+t} und danach referenziert \lpy{a} auf dieses Objekt.
Schauen wir uns dazu folgendes Beispiel an.
\begin{lstlisting}
thor = ['Valhalla', 42] # Erstellt thor
print(id(thor))         # 140541679765448
thor += ['Valgrind']    # fuegt 'Valgrind' hinzu
print(id(thor))         # 140541679765448
thor = thor + ['Suppe'] # erstellt neue Liste und nennt sie thor
print(id(thor))         # 140541679767112
\end{lstlisting}

Listen können auch scheibchenweise mit nur einer Operation verändert werden.
\begin{lstlisting}
a[i:j] = t   # Entferne den slice a[i:j] und fuege dort die Liste t ein
del a[i:j]   # Entferne den slice a[i:j]. Aequivalent zu a[i:j] = []
a[i:j:k] = t # Ersetze den slice a[i:j:k] durch die Elemente der Liste t
del a[i:j:k] # Entferne den slice a[i:j:k]
\end{lstlisting}

Außerdem können wir eine Liste leeren oder eine oberflächliche Kopie anfertigen.
Das funktioniert so:
\begin{lstlisting}
a.clear() # Entfernt alle Elemente von a
a.copy()  # Erstellt eine oberflaechliche Kopie von a
\end{lstlisting}
Eine oberflächliche Kopie erstellt eine neue Liste mit denselben Variablen.
Insbesondere sind die referenzierten Objekte der beiden Listen identisch.
Das will man oft, da so eine oberflaechliche Kopie sehr schnell erstellt ist.
Manchmal möchte man aber, dass die Objekte verschieden sind.
Wie das funktioniert, wird im Abschnitt \ref{section:std_data_types:sequenzen:kopien} besprochen.

Für Schleifendurchläufe (oder auch andere Situationen) kann es sinnvoll sein, eine Liste umzudrehen.
Außerdem möchte man Listen sortieren können, sofern das Sinn macht.
Dafür stehen diese beiden vielsagenden Funktionen bereit.
\begin{lstlisting}
a.reverse() # Kehrt a um
a.sort()    # Sortiert a, falls das Sinn macht
\end{lstlisting}
Genauer gesagt kann man eine Liste sortieren, sofern je zwei Elemente \lpy{x} und \lpy{y} der Liste mit \lpy{x < y} verglichen werden können.
Siehe hierzu auch \ref{section:klassen:spezielle_funktionen}.


\subsubsection{range}
\label{section:std_data_types:sequenzen:range}
Ein sogenanntes ``Rangeobjekt'' ist eine immutable Sequenz von aufsteigenden ganzen Zahlen, die sich so ähnlich wie ein Tupel verhält.
Es gibt drei einfache Möglichkeiten ein Rangeobjekt zu erstellen.
Um ein Rangeobjekt der Zahlen \lpy{0 <= i < n} zu erstellen, ruft man die Funktion \lpy{range(n)} auf.
Um ein Rangeobjekt der Zahlen \lpy{m <= i < n} zu erstellen, ruft man die Funktion \lpy{range(m,n)} auf.
Um ein Rangeobjekt der Zahlen \lpy{m <= i*k < n} zu erstellen, ruft man die Funktion \lpy{range(m,n,k)} auf.
\begin{lstlisting}
range(n)     # (0, 1, ..., n-1)
range(m,n)   # (m, m+1, ..., n-1
range(m,n,k) # (m, m+k, ...., m+l*k) mit m+l*k < k <= m+(l+1)*k
\end{lstlisting}
Für \lpy{n}, \lpy{m} und \lpy{k} sind auch negative Werte möglich, solange dabei eine endliche Sequenz von Zahlen erstellt werden kann.


\subsubsection{list comprehension}
\label{section:std_data_types:sequenzen:list_comprehension}
Die sogenannten \emph{list comprehension} ist eine sehr praktische Möglichkeit um aus bereits vorhandenen Sequenzen neue Listen zu erstellen.
Hier ist \lpy{a} eine Sequenz und wir wollen all diejenigen \lpy{x} aus \lpy{a} haben, die eine gewisse Bedingung \lpy{cond(x)} erfüllen.
\begin{lstlisting}
[ x for x in a if cond(x) ]
\end{lstlisting}
Außerdem können wir statt \lpy{x} eine (von \lpy{x}) abhängende) Expression \lpy{expr(x)} in unsere neue Liste stecken.
\begin{lstlisting}
[ expr(x) for x in a if cond(x) ]
\end{lstlisting}
Noch allgemeiner, können wir eine Expression \lpy{expra(x)} verwenden, wenn die Bedingung \lpy{cond(x)} erfüllt ist und im anderen Fall eine Expression \lpy{exprb(x)} verwenden.
\begin{lstlisting}
[ expra(x) if cond(x) else exprb(x) for x in a  ]
\end{lstlisting}

Hier drei Beispiele.
Wir wollen eine Liste aller natürlichen Zahlen $0 \le x < 9$, die nicht durch $4$ teilbar sind.
\begin{lstlisting}
[ x for x in range(10) if x % 4 != 0 ]
\end{lstlisting}
Wir bekommen ein Tupel \lpy{a} und wollen daraus eine Liste all seiner Strings erstellen.
\begin{lstlisting}
a = ('ss', 20 , 'rtt3', 50, 40.2, [344], ('akk', 'er') )
[ x for x in a if type(x) is str ]
\end{lstlisting}
Wir bekommen ein Tupel \lpy{a} der nur Strings enthält.
Wir wollen eine Liste, die alle Strings aus \lpy{a} enthält, außer der String ist \lpy{'Anakonda'}, den wollen wir durch \lpy{'Python'} ersetzen.
\begin{lstlisting}
a = ( 'blau', 'anakonda', 'Strasse', 'Anakonda', '42', 'Anakonda' )
[ x if x is not 'Anakonda' else 'Python' for x in a ]
\end{lstlisting}


\subsubsection{str}
\label{section:std_data_types:sequenzen:str}
Die immutable Sequenz \lpy{str} verfügt über sehr viele selbsterklärende Funktionen.
Zum Beispiel erstellt man aus einem String \lpy{s} mit \lpy{s.lower()} einen String, der nur aus Kleinbuchstaben besteht. 
Wir besprechen nun noch kurz die Funktion \lpy{format} und verweisen den interessierten Leser auf \cite[Library, Build-in Types, Text Sequence Type]{Python3}.

Oft macht es Sinn, einen String durch Einsetzung von Variablen zu erzeugen.
Dabei sollen die Variablen von der Beschreibung des zu entstehenden Strings getrennt sein.
Dieses Konzept sollte von der \C-Funktion \lcpp{printf} bekannt sein.
In \Python kann man das am einfachsten mit der Funktionen \lpy{s.format(...)} erreichen.
Dabei beschreibt \lpy{s} das Aussehen des Strings, also normalen Text sowie Platzhalter \lstinline[style=PyInline]|'{}'| und
die Parameter von \lpy{format} sind die Variablen, die in die Platzhalter eingesetzt werden.
Hier ein Beispiel:
\begin{lstlisting}
text = 'Hallo {}. Ich bin {} Jahre alt.'.format('Welt', 42)
print(text) # Hallo Welt. Ich bin 42 Jahre alt.
\end{lstlisting}

