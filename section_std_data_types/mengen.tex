\subsection{Mengen}
\label{section:std_data_types:mengen}
In Gegensatz zu Sequenzen, die wir im Abschnitt \ref{section:std_data_types:sequenzen} behandelt haben, sind Mengen ``ungeordnete Container''.
Ähnlich wie bei Cantor, verstehen wir unter einer Menge die Zusammenfassung von bestimmten, wohlunterschiedenen Objekten zu einem Ganzen.
Allerdings verhalten sich Mengen in \Python anders als ``mathematische Mengen'' (sie erfüllen zum Beispiel nicht das Extensionalitätsaxiom, i.e.\ sie sind nicht eindeutig durch ihre Elemente bestimmt).
Das spielt in der Praxis keine Rolle und ist gerüchten zu Folge nur für zwangsgestörte Möchtegernmathematiker ein Problem.


\subsubsection{Mengen erzeugen}
\label{section:std_data_types:mengen:mengen_erzeugen}
Eine Menge kann aus einer (möglicherweise) leeren Sequenz erzeugt werden, sofern jedes Element der Sequenz ``hashbar'' ist.
In Abschnitt \ref{section:klassen:hashbare_objekte} erklären wir genauer, was ``hashbare Objekte'' sind.
Für den Moment begnügen wir uns damit, dass die Standardtypen genau dann hashbar sind, wenn sie immutable sind.
Mengen selbst können entweder mutable oder immutable sein.
Die mutable Variante ist \lpy{set}, die immutable Variante ist \lpy{frozenset}.

Um eine Menge zu erzeugen, welche die Objekte \lpy{4}, \lpy{'Python'} und \lpy{None} zu enthält, rufen wir \lpy{set} mit einer Liste oder einem Tuple das diese Elemente enthält auf.
\begin{lstlisting}
s = set( [4, 'Python', None] )
\end{lstlisting}


\subsubsection{Operationen für set und frozenset}
\label{section:std_data_types:mengen:operationen_fuer_set_und_frozenset}
Folgende Operationen verändern die Menge nicht und stehen somit sowohl \lpy{set} als auch \lpy{frozenset} zur Verfügung.
\begin{lstlisting}
len(s)          # Gibt die Anzahl der Elemente zurueck
x in s          # True  gdw x in s vorkommt
x not in s      # False gdw x in s vorkommt
s == t          # True gdw ihre Elemente uebereinstimmen
s.isdisjoint(t) # True gdw s und t disjunkt sind
s <= t          # True gdw s eine Teilmenge von t ist
s < t           # True gdw s eine echte Teilmenge von t ist
s >= t          # True gdw t eine Teilmenge von s ist
s > t           # True gdw t eine echte Teilmenge von s ist
s | t | ...     # Die Vereinigung von s und t und ...
s & t & ...     # Der Schnitt von s und t und ...
s - t - ...     # Die Differenz von s und t und ...
s ^ t           # Die Symmetrische Differenz von s und t
s.copy()        # Erstelle eine oberflaechliche Kopie von s
\end{lstlisting}
Die Konzepte der oberflächlichen und tiefen Kopien sowie die daraus resultierenden Konsequenzen erklären wir in Abschnitt \ref{section:std_data_types:sequenzen:kopien} am Beispiel der Listen.
Analoge Aussagen gelten für Mengen.

An dieser Stelle gehen wir noch ein wenig auf die Gleichheit von Mengen ein.
Wie gesagt, wertet \lpy{s == t} genau dann als \lpy{True} aus, wenn die Elemente von \lpy{s} und \lpy{t} übereinstimmen.
Das heißt aber nicht, dass \lpy{s} und \lpy{t} dieselbe Identität haben, also an der selben Stelle im Speicher stehen.
Hier ein Beispiel.
\begin{lstlisting}
s = set( range(1,11,1) ) # Die Menge {1,2,...,10}
t = set( range(1,11,1) ) # Die Menge {1,2,...,10}
print( s == t )          # True
print( id(s) )           # 139836709444296
print( id(t) )           # 139836709038568
\end{lstlisting}


\subsubsection{Operationen für set}
\label{section:std_data_types:mengen:operationen_fuer_set}
Jede Menge \lpy{s} vom Typ \lpy{set} ist mutable.
Das heißt, \lpy{s} ist ein Objekt, der Wert von \lpy{s} ist die Menge seiner Elemente und wir können den Wert von \lpy{s} verändern.
Dazu stehen uns folgende Operationen bereit.
\begin{lstlisting}
s |= t | ... # Erweitert die Menge s um t, ...
s &= t & ... # Entfernt alle Elmente aus s, die nicht in t,... sind
s -= t | ... # Entfernt alle Elemente aus s, die in t, ... sind
s ^= t       # Entfernt alle Elemente aus s, die nicht gleichzeitig
             # in s und t sind
add(e)       # Erweitert s um e
remove(e)    # Entfernt e aus s; loest KeyError aus, wenn e nicht in s ist
discard(e)   # Entfernt e aus s falls e in s ist
pop()        # Entfernt ein Element aus s und gibt es zurueck
clear()      # Entfernt alle Elemente aus s
\end{lstlisting}

Wir bemerken hier nochmal, dass die Operationen nur den Wert der Menge \lpy{s} verändern, die Identität \lpy{id(s)} bleibt bestehen.
\begin{lstlisting}
s = set( range(1, 6,1) ) # Die Menge {1,2,...,5}
t = set( range(6,11,1) ) # Die Menge {5,6,...,10}
print( id(s) )           # 139836709038792
print( id(t) )           # 139836709444296
s.add(11)                # Fuege zu s das Element 11 hinzu
print( id(s) )           # 139836709038792
print( id(s |  t) )      # 139836709037448
s |= t                   # Fuege zu s die Elemente 5,6,...,10 hinzu
print( id(s) )           # 139836709038792
\end{lstlisting}
Insbesondere sind die Objekte \lpy{s |= t} und \lpy{s = s | t} verschieden.
