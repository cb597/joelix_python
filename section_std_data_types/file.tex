\subsection{Dateien}
\label{section:std_data_types:file}
In \Python ließt und schreibt man in Dateien ganz ähnlich wie in \CC.
Man öffnet eine Datei, ließt oder schreibt und schließt die Datei sobald man sie nicht mehr braucht.

Das Öffnen funktioniert durch Aufrufen der Funktion:
\begin{lstlisting}
open(file, mode='r')
\end{lstlisting}
Dabei ist \lpy{file} der Pfad der zu öffenden Datei und \lpy{mode} ist der sogenannte Modus.
Die wichtigsten Modi fassen wir hier zusammen.
\begin{lstlisting}
'r' # Lesenzugriff
'w' # Schreibzugriff: alter Inhalt wird ueberschrieben
'a' # Schreibzugriff: beginne am Ende der alten Datei
\end{lstlisting}
Will man Lese- und Schreibzugriff haben, hängt man an \lpy{'r'}, \lpy{'w'} oder \lpy{'a'} das Zeichen \lpy{'+'} an.
Also benutzen wir \lpy{'r+'} und \lpy{'w+'} für Lese- und Schreibzugriff, wobei wir am Anfang der Datei anfangen und
wir benutzen \lpy{'a+'} für Lese- und Schreibzugriff, wobei wir am Ende der Datei anfangen.

Eine Datei \lpy{datei} schließt man wie folgt:
\begin{lstlisting}
datei.close() # Schliesse die Datei.
\end{lstlisting}

Es gibt einfache Möglichkeiten um aus einer geöffneten Datei \lpy{datei} zu lesen.
Entweder man ließt sie zeilenweise aus \lpy{datei.readline(size=1)} oder man ließt sie zeichenweise aus \lpy{datei.read(size=1)}.
Insbesondere ließt \lpy{datei.readline()} eine Zeile und \lpy{datei.read()} ließt ein Zeichen.
Praktischerweise verhält sich eine geöffnete Datei wie eine Sequenz, das heißt, wir können sie ganz einfach mit in einer Schleife auslesen.
In folgendem lesen und drucken wir eine Datei zeilenweise aus.
\begin{lstlisting}
datei = open('dateiname.txt', 'r') # Oeffne die Datei 'dateiname.txt'
for zeile in datei:                # Fuer jede Zeile in der Datei
  print(zeile)                     # Drucke die Zeile aus
datei.close()                      # Schliesse die Datei.
\end{lstlisting}

In eine Datei schreiben wir wie folgt:
\begin{lstlisting}
datei.write(string)
datei.writelines(liste_von_strings)
\end{lstlisting}
