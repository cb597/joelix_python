\section{Ausnahmen}
\label{section:ausnahmen}
Gut geschriebene Bibliotheken und Programme stürzen nicht ab, wenn sie falsch bedient werden oder
wenn eine andere außergewöhnliche Situation eintritt.
Beispielsweise soll unsere \lcpp{joelixblas} Bibliothek nicht abstürzen, wenn der Benutzer eine Funktion mit unzulässigen Werten aufruft oder
wenn der Benutzer mehr Speicher verlangt, als das System uns bieten kann.

In \Python trennt man den ``gewöhnlichen Programmfluss'' und die ``außergewöhnlichen Programmfluss'' wie folgt:
Wenn der normale, sequenzielle Programmfluss durch ein außergewöhnliches Vorkommnis unterbrochen werden muss, nennt man das eine \emph{Ausnahme} oder \emph{Exception}.
Sobald eine Ausnahme \emph{ausgelöst} wird, muss sie \emph{behandelt} werden.

Typische Ausnahmen sind Fehler wie Syntaxfehler, Zugriffsfehler oder unzureichend viel Speicher.
Es gibt in \Python auch Ausnahmen, die nicht durch Fehler ausgelöst werden.
Außerdem kann man eigene Ausnahmen definieren.

Wird eine Ausnahme nicht behandelt, bricht der Programmfluss der gerade laufenden Funktion ab und die Ausnahme wird an die aufrufende Funktion weitergegeben.
Die Ausnahme wird dann entweder dort behandelt, oder das Weiterreichen wird fortgesetzt bis die Ausnahme entweder irgendwann behandelt wird oder das Programm mit der besagten Ausnahme abbricht.

Die folgende Ausnahme hat man bestimmt schon einige Male gesehen, wenn man Code direkt im \Python-Interpreter schreibt:
\begin{lstlisting}
a,b = 2,5
if a == b
  print( "A ist ja wirklich B")

# Der Code liefert:
#   File "...", line ...
#     if a == b
#            ^
# SyntaxError: invalid syntax
\end{lstlisting}
Hier wird die zweite Zeile vom \Python-Interpreter gelesen.
Dieser stellt einen Syntaxfehler fest und löst eine Ausnahme aus, die den Programmfluss an dieser Stelle unterbricht.
Schlussendlich wird das Programm beendet.

\subsection{Ausnahmen behandeln}
\label{section:ausnahmen:ausnahmen_behandeln}
Bevor wir erklären, wie eine Ausnahme behandelt wird, gehen wir kurz darauf ein, was eine Ausnahme ist:
Eine Ausnahme ist ein Objekt und somit hat sie einen festen Typ.
Der \Python-Standard definiert eine Hand voll Ausnahmetypen, wie zum Beispiel
den Ausnahmetyp \lpy{ZeroDivisionError} (der bei einer Division durch Null ausgelöst wird) oder
den Ausnahmetyp \lpy{SyntaxError} (den der \Python-Interpreter bei einem Syntaxfehler auslöst).
Der Wert einer Ausnahme \lpy{ausnahme} vom Ausnahmetyp \lpy{Ausnahmetyp} ist (für uns und nur hier) ein String und jede Ausnahme kann als String interpretiert werden.
Eine Liste der wichtigsten, bereits definierten Ausnahmetypen sind in Abschnitt \ref{section:ausnahmen:definierte_und_eigene_ausnahmen} zu finden.

Ein Programmabschnitt in dem Ausnahmen ausgelöst werden können, die wir (im Fall das Fälle) behandeln wollen,
wird der Programmabschnitt in ein \lpy{try-except} Konstrukt eingefasst.
Nach \lpy{try:} folgt unser (eingerückter) Programmabschnitt.
Dann werden die möglichen Ausnahmen behandelt.
Möchte oder muss man eine Ausnahme vom Ausnahmetyp \lpy{Ausnahmetyp} behandeln, geschieht das mit \lpy{except Ausnahmetyp:} gefolgt von der auszuführenden Ausnahmebehandlung.
Hier kann man auch mehrere Ausnahmentypen zusammenfassen mit \lpy{except (Ausnahmetyp_1, Ausnahmetyp_2, ...):}.
Alle anderen Ausnahmentypen sammelt man mit \lpy{except:}.
Das sieht dann zum Beispiel so aus:
\begin{lstlisting}
# Programmfluss

try:
  # Abschnitt der Ausnahmen ausloesen kann, die wir behandeln wollen
except Ausnahmetyp_1:
  # Ausnahmetyp 1 behandeln
except (Ausnahmetyp_2, Ausnahmetyp_3):
  # Ausnahmetyp 2 behandeln
# Hier noch mehr Ausnahmentypen die man behandeln moechte
except:
  # Alle anderen Ausnahmen behandeln

# Hier geht der normale Programmfluss weiter
\end{lstlisting}

Um auf eine Ausnahme von einem Ausnahmetyp zugreifen zu können nutzt man \lpy{except Ausnahmetyp as ausnahme}.
Hier ein Beispiel, das die Ausnahme ``Division durch Null'' behandelt.
\begin{lstlisting}
try:
  a = 1/0
except ZeroDivisionError as ausnahme:
  print('Ausnahme:', ausnahme)
\end{lstlisting}
Man kann auch mehrere Ausnahmetypen mit \lpy{as} benennen.
\begin{lstlisting}
try:
  a = 1/0
except (ZeroDivisionError, ValueError) as ausnahme:
  print('Ausnahme vom Typ:', type(ausnahme), 'mit Wert:', ausnahme )
\end{lstlisting}

Es gibt Situationen, da möchte man einen Programmabschnitt ausführen der Ausnahmen auslösen kann und diese dann folgendermaßen behandeln.
Wird eine (behandelnare) Ausnahme ausgelöst, so soll sie behandelt werden.
Wird jedoch keine Ausnahme ausgelöst, dann (und nur dann) soll ein weiterer Programmabschnitt ausgeführt werden.
Das funktioniert mit der \lpy{try-except-else} Konstruktion:
\begin{lstlisting}
# Programmfluss

try:
  # Abschnitt A
except ...: # Zu behandelnden Ausnahmetyp festlegen
  # Ausnahmen behandeln
... # Weitere Ausnahmebehandlungen
else: # Wird ausgefuehrt genau dann wenn Absch. A keine Ausnahme ausloest
  # Abschnitt B

# Hier geht der normale Programmfluss weiter
\end{lstlisting}
Der obige Code verhält sich genau wie der nachfolgende Code:
\begin{lstlisting}
try:
  ausnahme_aufgetreten = False
  # Abschnitt A
except ...: # Zu behandelnden Ausnahmetyp festlegen
  ausnahme_aufgetreten = True
  # Ausnahmen behandeln
... # Weitere Ausnahmebehandlungen, die ausnahme_aufgetreten=True setzen
if ausnahme_aufgetreten == False:
  # Abschnitt B
\end{lstlisting}

An dieser Stelle kann man bereits verstehen, warum Ausnahmen ein gutes Konzept sind.
Durch die Aufteilung in einen von \lpy{try} eingeleiteten Block schreibt man den auszuführenden Programmcode und
teilt die Ausnahmebehandlung in die von \lpy{except} eingeleiteten Blöcken ein.
Das führt zu wesentlich übersichtlicherem Code.

Hier noch ein Beispiel:
\begin{lstlisting}[escapechar=|]
import math
def ganzzahlige_wurzel( x ):
  """Diese Funktion zieht die ganzzahlige Wurzel."""
  y = 0 # Wir definieren eine Variable y, die wir am Ende zurueckgeben,
        # unabhaengig davon, ob eine Ausnahme behandelt werden muss oder
        # nicht
  try:  # Versuche die ganzzahlige Wurzel zu siehen
    y = int(math.sqrt(x)) |\label{zeile:sqrt_loest_fehler_aus}|
  except TypeError as ausnahme:  # Ausnahme: x hat den falschen Typ
    print('Falscher Typ:', ausnahme)
  except ValueError as ausnahme: # Ausnahme: x ist negativ.
    print('Falscher Wert:', ausnahme)
  except:                        # Ausnahme: andere Ausnahme
    print('Anderer, komischer Fehler...')
  return y |\label{zeile:y_ist_evtl_nicht_definiert}|

ganzzahlige_wurzel('Suppe') # Druckt: 'Falscher Typ: a float is required'
ganzzahlige_wurzel(-3)      # Druckt: 'Falscher Wert: math domain error'
ganzzahlige_wurzel(6)       # Druckt nix.
\end{lstlisting}
Man beachte, dass die Zeile \lpy{y=0} nicht vergessen werden darf, denn sonst kann es passieren, dass \lpy{y} in Zeile~\ref{zeile:y_ist_evtl_nicht_definiert} nicht definiert ist.
Falls \lpy{math.sqrt(x)} eine Ausnahme auslöst, wird der Programmfluss in Zeile~\ref{zeile:sqrt_loest_fehler_aus} unterbrochen und die Ausnahme behandelt.
Dass heißt, beim Auslösen einer Ausnahme wird \lpy{y} in dieser Zeile weder definiert noch auf ein Objekt gesetzt und kann insbesondere in Zeile~\ref{zeile:y_ist_evtl_nicht_definiert} nicht zurückgegeben werden.

\subsection{Ausnahmen auslösen und weitergeben}
\label{section:ausnahmen:ausnahmen_ausloesen}
Wir wollen nun verstehen, wie man Ausnahmen auslöst und wie Ausnahmen weitergegeben werden.
Beides geschieht mit \lpy{raise}.

Man löst eine Ausnahme vom Typ \lpy{Ausnahmetyp} mit dem beschreibenden String \lpy{ausnahmestring} durch folgendes Statement aus:
\begin{lstlisting}
raise Ausnahmetyp(ausnahmestring)
\end{lstlisting}
Wenn man sein Programm sehr trotzig abbrechen möchte kann man das also so tun:
\begin{lstlisting}
raise RuntimeError("Mir ist jetzt alles egal!")
\end{lstlisting}
Im folgenden Beispiel definieren wir eine Funktion, die nur mit Strings und Ganzzahlen umgehen möchte:
\begin{lstlisting}
def ich_mag_nur_strings_und_ganzzahlen( x ):
  """Diese Funktion mag nur Strings und Ganzzahlen."""
  if not (type(x) is int or type(x) is str):
    raise ValueError("Ich mag nur Strings und Ganzzahlen")
  print("Ich mag dich: '{}'".format(x))
\end{lstlisting}

Nun klären wir die Frage:
\begin{center}
  Wem wird eine Ausnahme zum Behandeln eigentlich übergeben?
\end{center}
Zuerst führen wir den sogenannten \emph{Call Tree} eines Programms ein.
Der Call Tree ist bei sequenziellen Programmen das ohne ausgelöse Ausnahmen auskommt immer ein gewurzelter Baum.
Die Wurzel $v$ ist die main-Funktion (oder genauer das main-Modul).
Wird eine Funktion \lpy{f} aufgerufen, definiert das eine Kante mit einem neuen Knoten in unserem Baum, den wir hier der Einfachheit halber $v(f)$ nennen.
Die in \lpy{f} aufgerufenen Funktionen, sagen wir \lpy{g}, \lpy{h} oder vielleicht sogar \lpy{f}, definieren dann neue Kanten zu neuen Knoten, sagen wir $v(g,f)$, $v(h,f)$, $v(f,f)$.
Wird eine Funktion \lpy{k} mehrere Male hintereinander aufgerufen, erstellen wir für jeden Aufruf eine neue Kante mit einem neuen Endknoten.

Per Konstruktion entspricht jede Kante einem Funktionsaufruf.
Braucht man $k$ Kanten um von der Wurzel $v$ zu einem anderen Knoten $w$ zu kommen, bedeutet dass wir $k$ ineinander verschachtelte Funktionsaufrufe benötigt haben.

Also entsteht zur Programmlaufzeit ein gewurzelter Baum.
Zu einem festen gewählen Zeitpunkt, während das Programm läuft, gibt es immer einen Knoten, der zuletzt erstellt wurde.
Diesen Knoten nennen wir ``aktiv''.
Der aktive Knoten entspricht dem Programmabschnitt, in dem wir uns (zur festgelegten Laufzeit) befinden.

Nun können wir leicht verstehen, wem eine Ausnahme zum Behandeln übergeben wird.
Dazu nehmen wir uns den Call Tree das laufende Programm zur Hilfe.
Wird eine Ausnahme in einem Programmabschnitt ausgelöst, so entspricht dieser Programmabschnitt dem aktiven Knoten.
Die Ausnahme kann dann vom aktiven Knoten behandelt werden, falls der Programmabschnitt im einem \lpy{try-except} Konstrukt liegt.
Außerdem muss die Ausnahme in ihrem \lpy{except} Abschnitt behandelt werden.
Ist mindestens eins von beiden nicht der Fall, wird die Ausnahme an den Knoten über dem aktiven Knoten zum Behandeln weitergeben.
Dies wird so lange fortgeführt, bis die Ausnahme behandelt wurde oder bis sie schlussendlich auch in der Wurzel nicht behandelt wurde.
Im letzteren Fall erklärt der \Python-Interpreter wo die Ausnahme aufgetreten ist und beendet das Programm.

Man kann sogar Ausnahmen behandeln und nach der Behanlung die Ausnahme mit \lpy{raise} an den darüberliegenden Knoten weitergeben.
Das besprechen wir an dieser Stelle aber nicht ausführlicher.

Schauen wir uns das ganze mal anhand eines Beispiels an:
\begin{lstlisting}
def drucke_float_aus(zahl): # Druckt float aus oder loest Ausnahme aus
  if type(zahl) is not float:
    raise TypeError('Ich will float und sonst nichts.')
  else:
    print('{}, ich liebe dich'.format(zahl))

def eins_durch_null(): # Teilt durch Null
  return (1.0/0.0)

def behandle_ausnahmen_nicht(): # Behandelt Ausnahmen nicht
  eins_durch_null()             # Ausnahme wird nicht behandelt

def behandle_ausnahmen():
  try:
    drucke_float_aus('Hallo!')
  except:
    print('Ausnahme an Stelle 1 wurde ausgeloest')
  
  try:
    drucke_float_aus(eins_durch_null())
  except:
    print('Ausnahme an Stelle 2 wurde ausgeloest')
  
  try:
    behandle_ausnahmen_nicht()
  except:
    print('Ausnahme an Stelle 3 wurde ausgeloest')
\end{lstlisting}
Beim Aufruf der Funktion \lpy{behandle_ausnahmen()} versuchen wir zunächst, die Funktion \lpy{drucke_float_aus('Hallo!')} auszuführen.
Diese löst eine Ausnahme aus, die wir in \lpy{behandle_ausnahmen()} abfangen.
In der Ausnahmebehandlung drucken wir:
\begin{center}
\lpy{'Ausnahme an Stelle 1 wurde ausgeloest'} 
\end{center}
Nun versuchen wir \lpy{drucke_float_aus(eins_durch_null())} auszuführen.
Dabei wird zuerst die innere Funktion, also \lpy{eins_durch_null()} ausgeführt.
Diese löst eine Ausnahme aus.
Insbesondere gibt die Funktion \lpy{eins_durch_null()} nichts zurück denn die normale Ausführung wird unterbrochen und wir machen sofort mit der Fehlerbehandlung weiter.
In der Ausnahmebehandlung drucken wir:
\begin{center}
\lpy{'Ausnahme an Stelle 2 wurde ausgeloest'}
\end{center}
Nun versuchen wir \lpy{behandle_ausnahmen_nicht()} auszuführen.
Diese Funktion ruft \lpy{eins_durch_null()} aus.
In unserem Call Tree haben wir momentan also einen Weg von \lpy{behandle_ausnahmen} über \lpy{behandle_ausnahmen_nicht} zu \lpy{eins_durch_null}.
Die Funktion \lpy{eins_durch_null} löst eine Ausnahme aus.
Diese wird an \lpy{behandle_ausnahmen_nicht} weitergegeben und dort nicht behandelt.
Also wird sie weitergegeben an \lpy{behandle_ausnahmen}.
Dort wird sie behandelt.
In der Ausnahmebehandlung drucken wir:
\begin{center}
\lpy{'Ausnahme an Stelle 3 wurde ausgeloest'}
\end{center}

\subsection{Bereits definierte und eigens definierte Ausnahmen}
\label{section:ausnahmen:definierte_und_eigene_ausnahmen}

Wir listen nachfolgend einige Ausnahmentypen auf.
Diese Liste ist nicht vollständig und wir verweisen den interessierten Leser auf \cite[Library: Exceptions]{Python3}.
Außerdem sind einige der Ausnahmen voneinander abgeleitet.
Da wir in diesem Kurs ``abgeleitete Klassen'' nicht behandelt haben, gehen wir hier weiter nicht darauf ein und verweisen nocheinmal auf \cite[Library: Exceptions]{Python3}.
\begin{lstlisting}
Exception          # Allgemeine Ausnahme.

FloatingPointError # Gleitkommafehler
OverflowError      # Overflowfehler
ZeroDivisionError  # Du hast durch Null geteilt
ImportError        # Fehler beim Importieren
IndexError         # Falscher Index beim Sequenzzugriff
KeyError           # Falscher Key beim Verzeichniszugriff
MemoryError        # Wir haben nicht genug Speicher
FileExistsError    # Datei existiert (beim erstellen einer neuen Datei)
FileNotFoundError  # Datei nicht gefunden (beim oeffnen einer Datei)
IsADirectoryError  # Dateioperation auf Ordner angewendet
NotADirectoryError # Ordneroperation auf Datei angewendet
PermissionError    # Unzureichende Zugriffsrechte (bei Dateien / Ordnern)
RuntimeError       # Laufzeitfehler (wird vom Programmierer ausgeloest)
NotImplementedError # Funktion ist nicht implementiert
SyntaxError        # Syntaxfehler
IndentationError   # Syntaxfehler: Falsch eingerueckt
SystemError        # Komischer Systemfehler
TypeError          # Falscher Typ
ValueError         # Falscher Wert


Warning            # Allgemeine Warnung

DeprecationWarning # Warnung: Veraltete Funktion / Klasse wird verwendet
ImportWarning      # Warnung beim Importieren
\end{lstlisting}

Da wir in diesem Kurs  ``abgeleitete Klassen'' nicht behandelt haben, erklären wir hier nur, wie man eigene Ausnahmen erstellt, aber nicht wie das im Detail funktioniert.
Eine eigenen Ausnahmetyp erstellt man so:
\begin{lstlisting}
class meine_ausnahme(Exception): pass
\end{lstlisting}
Dann kann man im \lpy{try-except} Konstrukt seinen eigenen Ausnahmetyp verwenden.
Hier ein Beispiel:
\begin{lstlisting}
import math

class NichtQuadratisch(Exception) : pass
class KeineReelenLoesungen(Exception) : pass

def loese_quad_gl(a,b,c):
  if a == 0:
    ausnbeschr = 'Nicht quadratisch a={}, b={}, c={}'.format(a,b,c)
    ausnahme = NichtQuadratisch(ausnbeschr)
    raise ausnahme
  if b**2 - 4.0*a*c < 0:
    ausnbeschr = 'Keine reelen Loesungen a={}, b={}, c={}'.format(a,b,c)
    ausnahme = KeineReelenLoesungen(ausnbeschr)
    raise ausnahme
  x1 = ( -b + math.sqrt( b**2 - 4.0*a*c) ) / (2.0*a)
  x2 = ( -b - math.sqrt( b**2 - 4.0*a*c) ) / (2.0*a)
  return x1, x2

try:
  loese_quad_gl(1,0,-1) # (1,0), (-1,0)
  loese_quad_gl(1,0,1)  # Ausnahme: Keine reelen Loesungen a=1, b=0, c=1
  loese_quad_gl(0,0,1)  # Ausnahme: Nicht quadratisch a=0, b=0, c=0
except (NichtQuadratisch, KeineReelenLoesungen) as ausnahme:
  print('Meine Ausnahme', ausnahme)
\end{lstlisting}

Als letztes erklären wir in diesem Abschnitt wie man in einem \lpy{try-except} Konstrukt die ``restlichen Ausnahmentypen'', also solche die mit \lpy{except:} behandelt werden, auch mit \lpy{as} benennen kann.
Warum das funktioniet erklären wir an dieser Stelle nicht, da man ``abgeleitete Klassen'' für die Erklärung braucht, die haben wir in diesem Kurs aber nicht behandeln können.
Um also ``die restlichen Ausnahmen'' mit \lpy{as} zu benennen verwendet man die Zeile \lpy{except Exception as ausnahme:}, auch wenn \lpy{Exception} nicht alle sondern nur alle sinnvollen Ausnahmen zusammenfasst.
Zum Beispiel ist die Ausnahme \lpy{SystemExit} nicht vom Typ \lpy{Exception}.
Die Ausnahmebehandlung soll so aussehen:
\begin{lstlisting}
try:
  # Programmabschnitt der Ausnahmen ausloesen kann
except Ausnahmetyp_1 as ausnahme:
  # Ausnahmetyp 1 behandeln
...
except Exception as ausnahme:
  # Alle anderen sinnvollen Ausnahmen behandeln
except:
  raise # Lass das mal lieber den Python Interpreter handhaben.
# Hier geht der normale Programmfluss weiter
\end{lstlisting}

