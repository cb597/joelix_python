\subsection{Ausnahmen auslösen und weitergeben}
\label{section:ausnahmen:ausnahmen_ausloesen}
Wir wollen nun verstehen, wie man Ausnahmen auslöst und wie Ausnahmen weitergegeben werden.
Beides geschieht mit \lpy{raise}.

Man löst eine Ausnahme vom Typ \lpy{Ausnahmetyp} mit dem beschreibenden String \lpy{ausnahmestring} durch folgendes Statement aus:
\begin{lstlisting}
raise Ausnahmetyp(ausnahmestring)
\end{lstlisting}
Wenn man sein Programm sehr trotzig abbrechen möchte kann man das also so tun:
\begin{lstlisting}
raise RuntimeError("Mir ist jetzt alles egal!")
\end{lstlisting}
Im folgenden Beispiel definieren wir eine Funktion, die nur mit Strings und Ganzzahlen umgehen möchte:
\begin{lstlisting}
def ich_mag_nur_strings_und_ganzzahlen( x ):
  """Diese Funktion mag nur Strings und Ganzzahlen."""
  if not (type(x) is int or type(x) is str):
    raise ValueError("Ich mag nur Strings und Ganzzahlen")
  print("Ich mag dich: '{}'".format(x))
\end{lstlisting}

Nun klären wir die Frage:
\begin{center}
  Wem wird eine Ausnahme zum Behandeln eigentlich übergeben?
\end{center}
Zuerst führen wir den sogenannten \emph{Call Tree} eines Programms ein.
Der Call Tree ist bei sequenziellen Programmen das ohne ausgelöse Ausnahmen auskommt immer ein gewurzelter Baum.
Die Wurzel $v$ ist die main-Funktion (oder genauer das main-Modul).
Wird eine Funktion \lpy{f} aufgerufen, definiert das eine Kante mit einem neuen Knoten in unserem Baum, den wir hier der Einfachheit halber $v(f)$ nennen.
Die in \lpy{f} aufgerufenen Funktionen, sagen wir \lpy{g}, \lpy{h} oder vielleicht sogar \lpy{f}, definieren dann neue Kanten zu neuen Knoten, sagen wir $v(g,f)$, $v(h,f)$, $v(f,f)$.
Wird eine Funktion \lpy{k} mehrere Male hintereinander aufgerufen, erstellen wir für jeden Aufruf eine neue Kante mit einem neuen Endknoten.

Per Konstruktion entspricht jede Kante einem Funktionsaufruf.
Braucht man $k$ Kanten um von der Wurzel $v$ zu einem anderen Knoten $w$ zu kommen, bedeutet dass wir $k$ ineinander verschachtelte Funktionsaufrufe benötigt haben.

Also entsteht zur Programmlaufzeit ein gewurzelter Baum.
Zu einem festen gewählen Zeitpunkt, während das Programm läuft, gibt es immer einen Knoten, der zuletzt erstellt wurde.
Diesen Knoten nennen wir ``aktiv''.
Der aktive Knoten entspricht dem Programmabschnitt, in dem wir uns (zur festgelegten Laufzeit) befinden.

Nun können wir leicht verstehen, wem eine Ausnahme zum Behandeln übergeben wird.
Dazu nehmen wir uns den Call Tree das laufende Programm zur Hilfe.
Wird eine Ausnahme in einem Programmabschnitt ausgelöst, so entspricht dieser Programmabschnitt dem aktiven Knoten.
Die Ausnahme kann dann vom aktiven Knoten behandelt werden, falls der Programmabschnitt im einem \lpy{try-except} Konstrukt liegt.
Außerdem muss die Ausnahme in ihrem \lpy{except} Abschnitt behandelt werden.
Ist mindestens eins von beiden nicht der Fall, wird die Ausnahme an den Knoten über dem aktiven Knoten zum Behandeln weitergeben.
Dies wird so lange fortgeführt, bis die Ausnahme behandelt wurde oder bis sie schlussendlich auch in der Wurzel nicht behandelt wurde.
Im letzteren Fall erklärt der \Python-Interpreter wo die Ausnahme aufgetreten ist und beendet das Programm.

Man kann sogar Ausnahmen behandeln und nach der Behanlung die Ausnahme mit \lpy{raise} an den darüberliegenden Knoten weitergeben.
Das besprechen wir an dieser Stelle aber nicht ausführlicher.

Schauen wir uns das ganze mal anhand eines Beispiels an:
\begin{lstlisting}
def drucke_float_aus(zahl): # Druckt float aus oder loest Ausnahme aus
  if type(zahl) is not float:
    raise TypeError('Ich will float und sonst nichts.')
  else:
    print('{}, ich liebe dich'.format(zahl))

def eins_durch_null(): # Teilt durch Null
  return (1.0/0.0)

def behandle_ausnahmen_nicht(): # Behandelt Ausnahmen nicht
  eins_durch_null()             # Ausnahme wird nicht behandelt

def behandle_ausnahmen():
  try:
    drucke_float_aus('Hallo!')
  except:
    print('Ausnahme an Stelle 1 wurde ausgeloest')
  
  try:
    drucke_float_aus(eins_durch_null())
  except:
    print('Ausnahme an Stelle 2 wurde ausgeloest')
  
  try:
    behandle_ausnahmen_nicht()
  except:
    print('Ausnahme an Stelle 3 wurde ausgeloest')
\end{lstlisting}
Beim Aufruf der Funktion \lpy{behandle_ausnahmen()} versuchen wir zunächst, die Funktion \lpy{drucke_float_aus('Hallo!')} auszuführen.
Diese löst eine Ausnahme aus, die wir in \lpy{behandle_ausnahmen()} abfangen.
In der Ausnahmebehandlung drucken wir:
\begin{center}
\lpy{'Ausnahme an Stelle 1 wurde ausgeloest'} 
\end{center}
Nun versuchen wir \lpy{drucke_float_aus(eins_durch_null())} auszuführen.
Dabei wird zuerst die innere Funktion, also \lpy{eins_durch_null()} ausgeführt.
Diese löst eine Ausnahme aus.
Insbesondere gibt die Funktion \lpy{eins_durch_null()} nichts zurück denn die normale Ausführung wird unterbrochen und wir machen sofort mit der Fehlerbehandlung weiter.
In der Ausnahmebehandlung drucken wir:
\begin{center}
\lpy{'Ausnahme an Stelle 2 wurde ausgeloest'}
\end{center}
Nun versuchen wir \lpy{behandle_ausnahmen_nicht()} auszuführen.
Diese Funktion ruft \lpy{eins_durch_null()} aus.
In unserem Call Tree haben wir momentan also einen Weg von \lpy{behandle_ausnahmen} über \lpy{behandle_ausnahmen_nicht} zu \lpy{eins_durch_null}.
Die Funktion \lpy{eins_durch_null} löst eine Ausnahme aus.
Diese wird an \lpy{behandle_ausnahmen_nicht} weitergegeben und dort nicht behandelt.
Also wird sie weitergegeben an \lpy{behandle_ausnahmen}.
Dort wird sie behandelt.
In der Ausnahmebehandlung drucken wir:
\begin{center}
\lpy{'Ausnahme an Stelle 3 wurde ausgeloest'}
\end{center}
