\subsection{Bereits definierte und eigens definierte Ausnahmen}
\label{section:ausnahmen:definierte_und_eigene_ausnahmen}

Wir listen nachfolgend einige Ausnahmentypen auf.
Diese Liste ist nicht vollständig und wir verweisen den interessierten Leser auf \cite[Library: Exceptions]{Python3}.
Außerdem sind einige der Ausnahmen voneinander abgeleitet.
Da wir in diesem Kurs ``abgeleitete Klassen'' nicht behandelt haben, gehen wir hier weiter nicht darauf ein und verweisen nocheinmal auf \cite[Library: Exceptions]{Python3}.
\begin{lstlisting}
Exception          # Allgemeine Ausnahme.

FloatingPointError # Gleitkommafehler
OverflowError      # Overflowfehler
ZeroDivisionError  # Du hast durch Null geteilt
ImportError        # Fehler beim Importieren
IndexError         # Falscher Index beim Sequenzzugriff
KeyError           # Falscher Key beim Verzeichniszugriff
MemoryError        # Wir haben nicht genug Speicher
FileExistsError    # Datei existiert (beim erstellen einer neuen Datei)
FileNotFoundError  # Datei nicht gefunden (beim oeffnen einer Datei)
IsADirectoryError  # Dateioperation auf Ordner angewendet
NotADirectoryError # Ordneroperation auf Datei angewendet
PermissionError    # Unzureichende Zugriffsrechte (bei Dateien / Ordnern)
RuntimeError       # Laufzeitfehler (wird vom Programmierer ausgeloest)
NotImplementedError # Funktion ist nicht implementiert
SyntaxError        # Syntaxfehler
IndentationError   # Syntaxfehler: Falsch eingerueckt
SystemError        # Komischer Systemfehler
TypeError          # Falscher Typ
ValueError         # Falscher Wert


Warning            # Allgemeine Warnung

DeprecationWarning # Warnung: Veraltete Funktion / Klasse wird verwendet
ImportWarning      # Warnung beim Importieren
\end{lstlisting}

Da wir in diesem Kurs  ``abgeleitete Klassen'' nicht behandelt haben, erklären wir hier nur, wie man eigene Ausnahmen erstellt, aber nicht wie das im Detail funktioniert.
Eine eigenen Ausnahmetyp erstellt man so:
\begin{lstlisting}
class meine_ausnahme(Exception): pass
\end{lstlisting}
Dann kann man im \lpy{try-except} Konstrukt seinen eigenen Ausnahmetyp verwenden.
Hier ein Beispiel:
\begin{lstlisting}
import math

class NichtQuadratisch(Exception) : pass
class KeineReelenLoesungen(Exception) : pass

def loese_quad_gl(a,b,c):
  if a == 0:
    ausnbeschr = 'Nicht quadratisch a={}, b={}, c={}'.format(a,b,c)
    ausnahme = NichtQuadratisch(ausnbeschr)
    raise ausnahme
  if b**2 - 4.0*a*c < 0:
    ausnbeschr = 'Keine reelen Loesungen a={}, b={}, c={}'.format(a,b,c)
    ausnahme = KeineReelenLoesungen(ausnbeschr)
    raise ausnahme
  x1 = ( -b + math.sqrt( b**2 - 4.0*a*c) ) / (2.0*a)
  x2 = ( -b - math.sqrt( b**2 - 4.0*a*c) ) / (2.0*a)
  return x1, x2

try:
  loese_quad_gl(1,0,-1) # (1,0), (-1,0)
  loese_quad_gl(1,0,1)  # Ausnahme: Keine reelen Loesungen a=1, b=0, c=1
  loese_quad_gl(0,0,1)  # Ausnahme: Nicht quadratisch a=0, b=0, c=0
except (NichtQuadratisch, KeineReelenLoesungen) as ausnahme:
  print('Meine Ausnahme', ausnahme)
\end{lstlisting}

Als letztes erklären wir in diesem Abschnitt wie man in einem \lpy{try-except} Konstrukt die ``restlichen Ausnahmentypen'', also solche die mit \lpy{except:} behandelt werden, auch mit \lpy{as} benennen kann.
Warum das funktioniet erklären wir an dieser Stelle nicht, da man ``abgeleitete Klassen'' für die Erklärung braucht, die haben wir in diesem Kurs aber nicht behandeln können.
Um also ``die restlichen Ausnahmen'' mit \lpy{as} zu benennen verwendet man die Zeile \lpy{except Exception as ausnahme:}, auch wenn \lpy{Exception} nicht alle sondern nur alle sinnvollen Ausnahmen zusammenfasst.
Zum Beispiel ist die Ausnahme \lpy{SystemExit} nicht vom Typ \lpy{Exception}.
Die Ausnahmebehandlung soll so aussehen:
\begin{lstlisting}
try:
  # Programmabschnitt der Ausnahmen ausloesen kann
except Ausnahmetyp_1 as ausnahme:
  # Ausnahmetyp 1 behandeln
...
except Exception as ausnahme:
  # Alle anderen sinnvollen Ausnahmen behandeln
except:
  raise # Lass das mal lieber den Python Interpreter handhaben.
# Hier geht der normale Programmfluss weiter
\end{lstlisting}
